\documentclass[12pt]{amsart}
%prepared in AMSLaTeX, under LaTeX2e


\usepackage{amssymb}

\theoremstyle{definition}
\newtheorem*{defn}{Definition}
\newtheorem*{summ}{Summary}

\theoremstyle{plain}
\newtheorem*{lem}{Lemma}
\newtheorem*{prop}{Proposition}
\newtheorem*{thm}{Theorem}

\theoremstyle{remark}
\newtheorem*{example}{Example}
\newtheorem*{remark}{Remark}

\newcommand{\CC}{\mathbb{C}}
\newcommand{\NN}{\mathbb{N}}
\newcommand{\QQ}{\mathbb{Q}}
\newcommand{\RR}{\mathbb{R}}
\newcommand{\ZZ}{\mathbb{Z}}

\newcommand{\eps}{\varepsilon}


\begin{document}

\section{Literature Review} % Main chapter title


\subsection{Survival Data} \hfill \\


As per \cite{kretowska}, we assume that the survival data is sampled from a random triple $(\mathbf{X}, T, \Delta)$. $\mathbf{X}$ is a real, $D$-dimensional random feature vector, i.e., $\mathbf{X} \in \RR^D$ for some integer $D$. $T = \min(T_0, C)$, where $T_0 \in \RR$ is the random variable indicating the survival time and $C \in \RR$ is random variable indicating the right-censoring time. $\Delta$ is a censoring indicator, i.e, $\Delta = I(T_0 < C)$. To construct a survival tree, we sample $N \in \ZZ_+$ observations $(\mathbf{x}_i, t_i, \delta_i)_{i = 1}^N$. 


\subsection{The Survival Dipolar Criterion} \hfill \\


In \cite{kretowska}, a non-parametric method of separating the feature space by oblique hyperplanes for prediction was proposed. She considered pairs of feature vectors $(\mathbf{x}_i, \mathbf{x}_j)$ (for integers $1 \leq i < j \leq N$), referred to as \emph{dipoles} in \cite{bobrowskikretowski}. She tagged each such pair as ``pure", ``mixed" or ``neither": ``pure" dipoles $(\mathbf{x}_i, \mathbf{x}_j)$ are those whose survival times $t_i, t_j$ are sufficiently close to one another; on the other hand, ``mixed" dipoles $(\mathbf{x}_i, \mathbf{x}_j)$ are those whose survival times $t_i, t_j$ are sufficiently far from one another.  Observations with inadequate survival information for dipolar classification are labeled as ``neither". \\

A non-parametric criterion to determine what constitutes ``sufficiently close" or ``sufficiently far" had already been suggested by \cite{bobrowskikretowski} for dipoles drawn from non-survival data. The main innovation in \cite{kretowska} was to extend this criterion to survival data using the time and censoring information $(t_i, \delta_i)_{i = 1}^N$. More concretely, she algorithmically constructs a sequence of survival time differences $\Delta T$ as follows:
\begin{align*}
&\Delta T \leftarrow () \quad \textit{\# start with the empty sequence} \\
&\text{ for integers } 1 \leq i < j \leq M: \\
&\quad \text{ if } (\delta_i = \delta_j = 1) \text{ then } \Delta T \leftarrow \Delta T.\text{append}(|t_i - t_j|) \\
&\quad \text{ if } (\delta_i = 0, \delta_j = 1 \text{ and } t_i > t_j) \text{ then } \Delta T \leftarrow \Delta T.\text{append}(t_i - t_j) \\
&\quad \text{ if } (\delta_i = 1, \delta_j = 0 \text{ and } t_j > t_i) \text{ then } \Delta T \leftarrow \Delta T.\text{append}(t_j - t_i)
\end{align*} Next, she considers the ordered statistics of $\Delta T$: $\Delta T_{(1)} \leq \cdots \leq \Delta T_{(L)}$ where the integer $L$ is the length of $\Delta T$. Then, \cite{kretowska} fixes real parameters $0 < \eta < \zeta < 1$ which will serve as the lower and upper percentile cutoffs for determining the ``pure" and ``mixed" dipoles. In other words, the floored products $\lfloor \eta \cdot K \rfloor$ and $\lfloor \zeta \cdot K \rfloor$ respectively approximate the $\eta$-th quantile and $\zeta$-th quantile of $\Delta T$. Using these, the dipolar criterion is defined as follows:
\begin{itemize}
	\item[1.] A dipole $(\mathbf{x}_i, \mathbf{x}_j)$ (for integers $1 \leq i < j \leq N$) is \emph{pure} if:
	
	$\delta_i = \delta_j = 1$ and $|t_i - t_j| < \Delta T_{(\lfloor \eta \cdot K \rfloor)}$
	
	\item[2.] A dipole $(\mathbf{x}_i, \mathbf{x}_j)$ (for integers $1 \leq i < j \leq N$) is \emph{mixed} if:
	
	$\delta_i = \delta_j = 1$ and $|t_i - t_j| \geq \Delta T_{(\lfloor \zeta \cdot K \rfloor)}$
	
	$\delta_i = 0, \delta_j = 1, t_i > t_j$ and $t_i - t_j \geq \Delta T_{(\lfloor \zeta \cdot K \rfloor)}$
	
	$\delta_i = 1, \delta_j = 0, t_i < t_j$ and $t_j - t_i \geq \Delta T_{(\lfloor \zeta \cdot K \rfloor)}$
	
	\item[3.] All other dipoles $(\mathbf{x}_i, \mathbf{x}_j)$ (for integers $1 \leq i < j \leq N$) are classified as ``neither" pure nor mixed.
\end{itemize}


\subsection{The Dipolar Criterion Function}


\subsubsection{Piecewise Linear Dipolar Criterion Functions} \hfill \\


Once the dipoles have been sorted into the aforementioned (disjoint) subsets, \cite{kretowska} considers the question of how to split the feature space $(\mathbf{x}_i)_{i = 1}^N$ using a hyperplane. Recall that a mixed dipole $(\mathbf{x}_i, \mathbf{x}_j)$ (for integers $1 \leq i < j \leq N$) describes points $\mathbf{x}_i$ and $\mathbf{x}_j$ that have different enough survival experience. On the other hand, a pure dipole $(\mathbf{x}_i, \mathbf{x}_j)$ (for integers $1 \leq i < j \leq N$) describes points $\mathbf{x}_i$ and $\mathbf{x}_j$ that have similar enough survival experience. Hence, the objective in \cite{kretowska} is find a hyperplane that splits as many mixed dipoles as possible without splitting too many pure dipoles. \\

\cite{kretowska} already suggested a way of achieving this using the following scheme.

\begin{itemize}
	\item First, note that any hyperplane in $\RR^D$ can be completely described by a $D + 1$-dimensional vector
	$$
	\mathbf{v} = \begin{pmatrix} -\theta \\ w_1 \\ \vdots \\ w_D \end{pmatrix}
	$$ This describes the hyperplane
	$
	w_1 x_1 + \cdots w_D x_D = \theta
	$ which can be succinctly written as the dot product
	$
	\mathbf{v}^T \mathbf{z} = \mathbf{z}^T \mathbf{v} = 0
	$ where 
	$$
	\mathbf{z} = \begin{pmatrix} 1 \\ x_1 \\ \vdots \\ x_D \end{pmatrix}
	$$ is the augmented vector of some $[x_1, \ldots, x_D]^T$ in $\RR^D$. Using this notation, we can let
	$
	\mathbf{z}_j = \begin{pmatrix} 1 \\ \mathbf{x}_j \end{pmatrix} 
	$ be the \emph{augmented vector} of each feature vector $\mathbf{x}_j$, and similarly we can let $
	(\mathbf{z}_j, \mathbf{z}_k 
	$ be the \emph{augmented dipole} of each dipole $(\mathbf{x}_j, \mathbf{x}_k)$ ($1 \leq j < k \leq N$).
	\item Next, using the above terminology, \cite{bobrowskikretowski} defined the following piecewise linear functions $\varphi^+_j, \varphi^-_j : \RR^{D + 1} \to \RR$ for each integer $1 \leq j \leq N$:
	\begin{align*}
		\varphi^+_j(\mathbf{v}) &= \max\{0, \eps_j - \mathbf{v}^T \mathbf{z}_j\} \quad \mathbf{v} \in \RR^{D + 1} \\
		\varphi^-_j(\mathbf{v}) &= \max\{0, \eps_j + \mathbf{v}^T \mathbf{z}_j\} \quad \mathbf{v} \in \RR^{D + 1}
	\end{align*} where the $\eps_j > 0$ are preset parameters usually set to $1$.
\end{itemize}

Now, we can use simple linear combinations of these functions to define penalty functions that split mixed dipoles when they are minimized. Indeed, given a mixed dipole $(\mathbf{x}_j, \mathbf{x}_k)$ ($1 \leq j < k \leq N$) note that either of the following combinations
\begin{align*}
	\varphi^{m^+}_{jk} &= \varphi^+_j + \varphi^-_k \text{ or } \\
	\varphi^{m^-}_{jk} &= \varphi^-_j + \varphi^+_k
\end{align*} can be used to split the mixed dipole. \\

Similarly, given a pure dipole $(\mathbf{x}_j, \mathbf{x}_k)$ ($1 \leq j < k \leq N$), either of the following combinations can be used to define penalty functions whose minimizer(s) do not split the pure dipole:
\begin{align*}
	\varphi^{p^+}_{jk} &= \varphi^+_j + \varphi^+_k \text{ or } \\
	\varphi^{p^-}_{jk} &= \varphi^-_j + \varphi^-_k
\end{align*} \\


\subsubsection{Heuristic Criterion for Dipolar Orientation} \hfill \\


Note that in each case we get two choices of penalty functions --- $\varphi^{m^+}_{jk}$ vs. $\varphi^{m^-}_{jk}$ for the mixed case --- and --- $\varphi^{p^+}_{jk}$ vs. $\varphi^{p^-}_{jk}$ for the pure case --- that can be used to split or not split some dipole $(\mathbf{x}_j, \mathbf{x}_k)$ ($1 \leq j < k \leq N$). For each case in general, the resulting the minimizers arising from the two choices can be different. Since we want a single optimal hyperplane to split the whole feature space, we need to find a systematic way of choosing among the pairs of choices above. \\

As a first step towards this, \cite{kretowska} heuristically defines the \emph{orientation} of an augmented dipole $(\mathbf{z}_j, \mathbf{z}_k)$ ($1 \leq j < k \leq N$) relative to some optimal hyperplane $\mathbf{v}^\ast \in \RR^{D + 1}$.

\begin{itemize}
	\item A mixed dipole $(\mathbf{z}_j, \mathbf{z}_k)$ ($1 \leq j < k \leq N$) is said to have a \emph{positive orientation} if we ``expect" \cite{kretowska} to have $(\mathbf{v}^\ast)^T \mathbf{z}_j \geq 0$ and $(\mathbf{v}^\ast)^T \mathbf{z}_k \leq 0$. Similarly it is said to have a \emph{negative orientation} if we ``expect" \cite{kretowska} to have $(\mathbf{v}^\ast)^T \mathbf{z}_j \leq 0$ and $(\mathbf{v}^\ast)^T \mathbf{z}_k \geq 0$. \\
	
	In the first case of a positive orientation, \cite{kretowska} chooses $\varphi^{m^+}_{jk}$ for the mixed dipole $(\mathbf{z}_j, \mathbf{z}_k)$ whereas in the second case of a negative orientation  \cite{kretowska} chooses $\varphi^{m^-}_{jk}$ for the mixed dipole $(\mathbf{z}_j, \mathbf{z}_k)$. \\
	
	\item A pure dipole $(\mathbf{z}_j, \mathbf{z}_k)$ ($1 \leq j < k \leq N$) is said to have a \emph{positive orientation} if we ``expect" \cite{kretowska} to have $(\mathbf{v}^\ast)^T \mathbf{z}_j \geq 0$ and $(\mathbf{v}^\ast)^T \mathbf{z}_k \geq 0$. Similarly it is said to have a \emph{negative orientation} if we ``expect" \cite{kretowska} to have $(\mathbf{v}^\ast)^T \mathbf{z}_j \leq 0$ and $(\mathbf{v}^\ast)^T \mathbf{z}_k \leq 0$. \\
	
	In the first case of a positive orientation, \cite{kretowska} chooses $\varphi^{p^+}_{jk}$ for the pure dipole $(\mathbf{z}_j, \mathbf{z}_k)$ whereas in the second case of a negative orientation  \cite{kretowska} chooses $\varphi^{p^-}_{jk}$ for the pure dipole $(\mathbf{z}_j, \mathbf{z}_k)$. \\
\end{itemize} Notice that not all cases are covered above. For instance, it is entirely possible for a mixed dipole $(\mathbf{z}_j, \mathbf{z}_k)$ to end up in the case where both $\mathbf{z}_j$ and $\mathbf{z}_k$ fall on the same side the optimal hyperplane $\mathbf{v}^\ast$. And such cases are not defined by \cite{kretowska} above. \cite{kretowska} does not elaborate further on a more exhaustive definition of dipole orientation. Hence, in Chapter 3, an original exhaustive definition of dipole orientation is given.


\subsubsection{The Dipolar Criterion Function} \hfill \\


Assuming for now that some guess of an optimal hyperplane $\mathbf{v}^\ast \in \RR^{D + 1}$ is fixed and that we have successfully chosen among the pairs of penalty functions for each of the mixed and pure dipoles, \cite{kretowska} continues on to define the \emph{dipolar criterion function}.


\bibliographystyle{unsrt}
\bibliography{refs}



% EXCEPT LEAVE THIS:
\end{document}

\documentclass[12pt]{amsart}
%prepared in AMSLaTeX, under LaTeX2e


\usepackage{amssymb}

\theoremstyle{definition}
\newtheorem*{defn}{Definition}
\newtheorem*{summ}{Summary}

\theoremstyle{plain}
\newtheorem*{lem}{Lemma}
\newtheorem*{prop}{Proposition}
\newtheorem*{thm}{Theorem}

\theoremstyle{remark}
\newtheorem*{example}{Example}
\newtheorem*{remark}{Remark}

\newcommand{\CC}{\mathbb{C}}
\newcommand{\NN}{\mathbb{N}}
\newcommand{\QQ}{\mathbb{Q}}
\newcommand{\RR}{\mathbb{R}}
\newcommand{\ZZ}{\mathbb{Z}}

\newcommand{\eps}{\varepsilon}


\begin{document}


\section{Induction of Survival Trees using the Dipolar Criterion Function}


\subsection{Precise Criterion for Dipolar Orientation} \hfill \\

Recall that the criterion for dipolar orientation suggested by \cite{kretowska} was non-exhaustive. As such, we provide here a precise criterion to determine dipolar orientation in all cases. \\

A dipole (mixed or pure) $(\mathbf{z}_j, \mathbf{z}_k)$ ($1 \leq j < k \leq N$) is said to have a \emph{positive orientation} if we have ${\mathbf{v}^{\ast}}^T (\mathbf{z}_j - \mathbf{z}_k) \geq 0$. Similarly it is said to have a \emph{negative orientation} if we have ${{\mathbf{v}^\ast}}^T (\mathbf{z}_j - \mathbf{z}_k) \leq 0$. \\
	
\begin{itemize}

	\item Now if $(\mathbf{z}_j, \mathbf{z}_k)$ ($1 \leq j < k \leq N$) is mixed with positive orientation, we choose $\varphi^{m^+}_{jk}$ for it whereas if it has negative orientation  we choose $\varphi^{m^-}_{jk}$ for it. \\
	
	\item And if $(\mathbf{z}_j, \mathbf{z}_k)$ ($1 \leq j < k \leq N$) is pure with positive orientation, we choose $\varphi^{p^+}_{jk}$ for it whereas if it has negative orientation  we choose $\varphi^{p^-}_{jk}$ for it. \\
	
\end{itemize}



\bibliographystyle{unsrt}
\bibliography{refs}



% EXCEPT LEAVE THIS:
\end{document}

\documentclass[]{article}
\usepackage{lmodern}
\usepackage{amssymb,amsmath}
\usepackage{ifxetex,ifluatex}
\usepackage{fixltx2e} % provides \textsubscript
\ifnum 0\ifxetex 1\fi\ifluatex 1\fi=0 % if pdftex
  \usepackage[T1]{fontenc}
  \usepackage[utf8]{inputenc}
\else % if luatex or xelatex
  \ifxetex
    \usepackage{mathspec}
  \else
    \usepackage{fontspec}
  \fi
  \defaultfontfeatures{Ligatures=TeX,Scale=MatchLowercase}
\fi
% use upquote if available, for straight quotes in verbatim environments
\IfFileExists{upquote.sty}{\usepackage{upquote}}{}
% use microtype if available
\IfFileExists{microtype.sty}{%
\usepackage{microtype}
\UseMicrotypeSet[protrusion]{basicmath} % disable protrusion for tt fonts
}{}
\usepackage[margin=1in]{geometry}
\usepackage{hyperref}
\hypersetup{unicode=true,
            pdftitle={MATH 629 HW 5 Solutions},
            pdfauthor={Drew Lazar},
            pdfborder={0 0 0},
            breaklinks=true}
\urlstyle{same}  % don't use monospace font for urls
\usepackage{color}
\usepackage{fancyvrb}
\newcommand{\VerbBar}{|}
\newcommand{\VERB}{\Verb[commandchars=\\\{\}]}
\DefineVerbatimEnvironment{Highlighting}{Verbatim}{commandchars=\\\{\}}
% Add ',fontsize=\small' for more characters per line
\usepackage{framed}
\definecolor{shadecolor}{RGB}{248,248,248}
\newenvironment{Shaded}{\begin{snugshade}}{\end{snugshade}}
\newcommand{\AlertTok}[1]{\textcolor[rgb]{0.94,0.16,0.16}{#1}}
\newcommand{\AnnotationTok}[1]{\textcolor[rgb]{0.56,0.35,0.01}{\textbf{\textit{#1}}}}
\newcommand{\AttributeTok}[1]{\textcolor[rgb]{0.77,0.63,0.00}{#1}}
\newcommand{\BaseNTok}[1]{\textcolor[rgb]{0.00,0.00,0.81}{#1}}
\newcommand{\BuiltInTok}[1]{#1}
\newcommand{\CharTok}[1]{\textcolor[rgb]{0.31,0.60,0.02}{#1}}
\newcommand{\CommentTok}[1]{\textcolor[rgb]{0.56,0.35,0.01}{\textit{#1}}}
\newcommand{\CommentVarTok}[1]{\textcolor[rgb]{0.56,0.35,0.01}{\textbf{\textit{#1}}}}
\newcommand{\ConstantTok}[1]{\textcolor[rgb]{0.00,0.00,0.00}{#1}}
\newcommand{\ControlFlowTok}[1]{\textcolor[rgb]{0.13,0.29,0.53}{\textbf{#1}}}
\newcommand{\DataTypeTok}[1]{\textcolor[rgb]{0.13,0.29,0.53}{#1}}
\newcommand{\DecValTok}[1]{\textcolor[rgb]{0.00,0.00,0.81}{#1}}
\newcommand{\DocumentationTok}[1]{\textcolor[rgb]{0.56,0.35,0.01}{\textbf{\textit{#1}}}}
\newcommand{\ErrorTok}[1]{\textcolor[rgb]{0.64,0.00,0.00}{\textbf{#1}}}
\newcommand{\ExtensionTok}[1]{#1}
\newcommand{\FloatTok}[1]{\textcolor[rgb]{0.00,0.00,0.81}{#1}}
\newcommand{\FunctionTok}[1]{\textcolor[rgb]{0.00,0.00,0.00}{#1}}
\newcommand{\ImportTok}[1]{#1}
\newcommand{\InformationTok}[1]{\textcolor[rgb]{0.56,0.35,0.01}{\textbf{\textit{#1}}}}
\newcommand{\KeywordTok}[1]{\textcolor[rgb]{0.13,0.29,0.53}{\textbf{#1}}}
\newcommand{\NormalTok}[1]{#1}
\newcommand{\OperatorTok}[1]{\textcolor[rgb]{0.81,0.36,0.00}{\textbf{#1}}}
\newcommand{\OtherTok}[1]{\textcolor[rgb]{0.56,0.35,0.01}{#1}}
\newcommand{\PreprocessorTok}[1]{\textcolor[rgb]{0.56,0.35,0.01}{\textit{#1}}}
\newcommand{\RegionMarkerTok}[1]{#1}
\newcommand{\SpecialCharTok}[1]{\textcolor[rgb]{0.00,0.00,0.00}{#1}}
\newcommand{\SpecialStringTok}[1]{\textcolor[rgb]{0.31,0.60,0.02}{#1}}
\newcommand{\StringTok}[1]{\textcolor[rgb]{0.31,0.60,0.02}{#1}}
\newcommand{\VariableTok}[1]{\textcolor[rgb]{0.00,0.00,0.00}{#1}}
\newcommand{\VerbatimStringTok}[1]{\textcolor[rgb]{0.31,0.60,0.02}{#1}}
\newcommand{\WarningTok}[1]{\textcolor[rgb]{0.56,0.35,0.01}{\textbf{\textit{#1}}}}
\usepackage{graphicx,grffile}
\makeatletter
\def\maxwidth{\ifdim\Gin@nat@width>\linewidth\linewidth\else\Gin@nat@width\fi}
\def\maxheight{\ifdim\Gin@nat@height>\textheight\textheight\else\Gin@nat@height\fi}
\makeatother
% Scale images if necessary, so that they will not overflow the page
% margins by default, and it is still possible to overwrite the defaults
% using explicit options in \includegraphics[width, height, ...]{}
\setkeys{Gin}{width=\maxwidth,height=\maxheight,keepaspectratio}
\IfFileExists{parskip.sty}{%
\usepackage{parskip}
}{% else
\setlength{\parindent}{0pt}
\setlength{\parskip}{6pt plus 2pt minus 1pt}
}
\setlength{\emergencystretch}{3em}  % prevent overfull lines
\providecommand{\tightlist}{%
  \setlength{\itemsep}{0pt}\setlength{\parskip}{0pt}}
\setcounter{secnumdepth}{0}
% Redefines (sub)paragraphs to behave more like sections
\ifx\paragraph\undefined\else
\let\oldparagraph\paragraph
\renewcommand{\paragraph}[1]{\oldparagraph{#1}\mbox{}}
\fi
\ifx\subparagraph\undefined\else
\let\oldsubparagraph\subparagraph
\renewcommand{\subparagraph}[1]{\oldsubparagraph{#1}\mbox{}}
\fi

%%% Use protect on footnotes to avoid problems with footnotes in titles
\let\rmarkdownfootnote\footnote%
\def\footnote{\protect\rmarkdownfootnote}

%%% Change title format to be more compact
\usepackage{titling}

% Create subtitle command for use in maketitle
\providecommand{\subtitle}[1]{
  \posttitle{
    \begin{center}\large#1\end{center}
    }
}

\setlength{\droptitle}{-2em}

  \title{MATH 629 HW 5 Solutions}
    \pretitle{\vspace{\droptitle}\centering\huge}
  \posttitle{\par}
    \author{Drew Lazar}
    \preauthor{\centering\large\emph}
  \postauthor{\par}
    \date{}
    \predate{}\postdate{}
  

\begin{document}
\maketitle

\hypertarget{cleaning-up-and-loading-necessary-packages}{%
\subsection{Cleaning up and loading necessary
packages}\label{cleaning-up-and-loading-necessary-packages}}

\begin{Shaded}
\begin{Highlighting}[]
\KeywordTok{rm}\NormalTok{(}\DataTypeTok{list=}\KeywordTok{ls}\NormalTok{())}
\KeywordTok{library}\NormalTok{(survival)}
\end{Highlighting}
\end{Shaded}

\begin{verbatim}
## Warning: package 'survival' was built under R version 3.6.3
\end{verbatim}

\begin{Shaded}
\begin{Highlighting}[]
\KeywordTok{library}\NormalTok{(dplyr)}
\end{Highlighting}
\end{Shaded}

\begin{verbatim}
## 
## Attaching package: 'dplyr'
\end{verbatim}

\begin{verbatim}
## The following objects are masked from 'package:stats':
## 
##     filter, lag
\end{verbatim}

\begin{verbatim}
## The following objects are masked from 'package:base':
## 
##     intersect, setdiff, setequal, union
\end{verbatim}

\hypertarget{loading-the-data}{%
\subsection{1. loading the data}\label{loading-the-data}}

\begin{Shaded}
\begin{Highlighting}[]
\NormalTok{Ven.reset <-}\KeywordTok{read.csv}\NormalTok{(}\StringTok{"VenresetMft.csv"}\NormalTok{, }\DataTypeTok{header =} \OtherTok{TRUE}\NormalTok{)}
\end{Highlighting}
\end{Shaded}

\hypertarget{i.-test-appropriateness-of-the-weibull-model-with-a-fixed-shape-parameter-p.}{%
\subsection{2i. Test appropriateness of the Weibull model with a fixed
shape parameter
p.}\label{i.-test-appropriateness-of-the-weibull-model-with-a-fixed-shape-parameter-p.}}

\#Create a survival object

\begin{Shaded}
\begin{Highlighting}[]
\NormalTok{Y<-}\KeywordTok{Surv}\NormalTok{(Ven.reset}\OperatorTok{$}\NormalTok{eventtime,Ven.reset}\OperatorTok{$}\NormalTok{status}\OperatorTok{==}\DecValTok{1}\NormalTok{)}
\end{Highlighting}
\end{Shaded}

\begin{Shaded}
\begin{Highlighting}[]
\CommentTok{#Partition LO2 into four strata and create KM estimators for each strata  }
\KeywordTok{quantile}\NormalTok{(Ven.reset}\OperatorTok{$}\NormalTok{LO2)}
\end{Highlighting}
\end{Shaded}

\begin{verbatim}
##      0%     25%     50%     75%    100% 
## -2.5600  1.0075  2.0350  3.0825  6.2300
\end{verbatim}

\begin{Shaded}
\begin{Highlighting}[]
\NormalTok{Ven.reset}\OperatorTok{$}\NormalTok{LO2.group<-}\KeywordTok{cut}\NormalTok{(Ven.reset}\OperatorTok{$}\NormalTok{LO2,}\KeywordTok{c}\NormalTok{(}\OperatorTok{-}\FloatTok{2.58}\NormalTok{,}\FloatTok{1.0075}\NormalTok{,}\FloatTok{2.0350}\NormalTok{,}\FloatTok{3.0825}\NormalTok{,}\FloatTok{6.2300}\NormalTok{),}\DataTypeTok{labels=}\KeywordTok{c}\NormalTok{(}\StringTok{'1'}\NormalTok{,}\StringTok{'2'}\NormalTok{,}\StringTok{'3'}\NormalTok{,}\StringTok{'4'}\NormalTok{))}
\NormalTok{kmfitO24=}\KeywordTok{survfit}\NormalTok{(Y}\OperatorTok{~}\NormalTok{Ven.reset}\OperatorTok{$}\NormalTok{LO2.group)}
\end{Highlighting}
\end{Shaded}

\begin{Shaded}
\begin{Highlighting}[]
\CommentTok{#Plot log-log curves against ln(t) for LO2.group}
\KeywordTok{plot}\NormalTok{(kmfitO24,}\DataTypeTok{fun=}\StringTok{"cloglog"}\NormalTok{,}\DataTypeTok{xlab=}\StringTok{"time in days on log scale"}\NormalTok{,}\DataTypeTok{ylab=}\StringTok{"log-log survival"}\NormalTok{, }\DataTypeTok{main=}\StringTok{"log-log curves by LO2group"}\NormalTok{,}\DataTypeTok{col=}\KeywordTok{c}\NormalTok{(}\StringTok{"red"}\NormalTok{,}\StringTok{"green"}\NormalTok{,}\StringTok{"blue"}\NormalTok{,}\StringTok{"black"}\NormalTok{))}
\KeywordTok{legend}\NormalTok{(}\StringTok{"topleft"}\NormalTok{,}\DataTypeTok{cex=}\NormalTok{.}\DecValTok{7}\NormalTok{,}\KeywordTok{c}\NormalTok{(}\StringTok{"LO2=high"}\NormalTok{,}\StringTok{"LO2=medhigh"}\NormalTok{,}\StringTok{"LO2=medlow"}\NormalTok{,}\StringTok{"LO2=low"}\NormalTok{),}\DataTypeTok{lty=}\KeywordTok{c}\NormalTok{(}\StringTok{"solid"}\NormalTok{),}\DataTypeTok{col=}\KeywordTok{c}\NormalTok{(}\StringTok{"red"}\NormalTok{,}\StringTok{"green"}\NormalTok{,}\StringTok{"blue"}\NormalTok{,}\StringTok{"black"}\NormalTok{))}
\end{Highlighting}
\end{Shaded}

\includegraphics{HW5_629Solutions_files/figure-latex/unnamed-chunk-6-1.pdf}

\begin{Shaded}
\begin{Highlighting}[]
\NormalTok{kmfitST3=}\KeywordTok{survfit}\NormalTok{(Y}\OperatorTok{~}\NormalTok{Ven.reset}\OperatorTok{$}\NormalTok{Setting)}
\KeywordTok{plot}\NormalTok{(kmfitST3,}\DataTypeTok{fun=}\StringTok{"cloglog"}\NormalTok{,}\DataTypeTok{xlab=}\StringTok{"time in days on log scale"}\NormalTok{,}\DataTypeTok{ylab=}\StringTok{"log-log survival"}\NormalTok{, }\DataTypeTok{main=}\StringTok{"log-log curves by Setting"}\NormalTok{,}\DataTypeTok{col=}\KeywordTok{c}\NormalTok{(}\StringTok{'red'}\NormalTok{,}\StringTok{'green'}\NormalTok{,}\StringTok{'blue'}\NormalTok{))}
\KeywordTok{legend}\NormalTok{(}\StringTok{"topleft"}\NormalTok{,}\KeywordTok{c}\NormalTok{(}\StringTok{"Setting=0"}\NormalTok{,}\StringTok{"Setting=1"}\NormalTok{,}\StringTok{"Setting=2"}\NormalTok{),}\DataTypeTok{lty=}\KeywordTok{c}\NormalTok{(}\StringTok{"solid"}\NormalTok{),}\DataTypeTok{col=}\KeywordTok{c}\NormalTok{(}\StringTok{"red"}\NormalTok{,}\StringTok{"green"}\NormalTok{,}\StringTok{"blue"}\NormalTok{))}
\end{Highlighting}
\end{Shaded}

\includegraphics{HW5_629Solutions_files/figure-latex/unnamed-chunk-7-1.pdf}
\textbf{Comment:} The log-log plot for LO2.group against ln(t) and the
log-log plot for Setting against ln(t) are both roughly straight and
parallel which suggests they both individually satisfy the Weibull model
with fixed \(p\).

\hypertarget{ii.}{%
\subsection{2ii.}\label{ii.}}

\begin{Shaded}
\begin{Highlighting}[]
\CommentTok{#Stratify Lo2 and Setting into six strata }
\NormalTok{Ven.reset}\OperatorTok{$}\NormalTok{LO2.group2<-}\KeywordTok{cut}\NormalTok{(Ven.reset}\OperatorTok{$}\NormalTok{LO2,}\KeywordTok{c}\NormalTok{(}\OperatorTok{-}\FloatTok{2.58}\NormalTok{,}\FloatTok{1.9}\NormalTok{,}\FloatTok{6.23}\NormalTok{),}\DataTypeTok{labels=}\KeywordTok{c}\NormalTok{(}\StringTok{'1'}\NormalTok{,}\StringTok{'2'}\NormalTok{))}
\NormalTok{Ven.reset}\OperatorTok{$}\NormalTok{LO2.Set.Strata<-}\KeywordTok{rep}\NormalTok{(}\DecValTok{0}\NormalTok{,}\KeywordTok{nrow}\NormalTok{(Ven.reset))}
\ControlFlowTok{for}\NormalTok{ (i }\ControlFlowTok{in} \DecValTok{1}\OperatorTok{:}\KeywordTok{nrow}\NormalTok{(Ven.reset))}
\NormalTok{\{}
  \ControlFlowTok{if}\NormalTok{ (Ven.reset}\OperatorTok{$}\NormalTok{LO2.group2[i]}\OperatorTok{==}\DecValTok{1} \OperatorTok{&}\StringTok{ }\NormalTok{Ven.reset}\OperatorTok{$}\NormalTok{Setting[i] }\OperatorTok{==}\DecValTok{0}\NormalTok{)\{}
\NormalTok{    Ven.reset}\OperatorTok{$}\NormalTok{LO2.Set.Strata[i]=}\DecValTok{1}
\NormalTok{  \} }\ControlFlowTok{else} \ControlFlowTok{if}\NormalTok{ (Ven.reset}\OperatorTok{$}\NormalTok{LO2.group2[i]}\OperatorTok{==}\DecValTok{2} \OperatorTok{&}\StringTok{ }\NormalTok{Ven.reset}\OperatorTok{$}\NormalTok{Setting[i] }\OperatorTok{==}\DecValTok{0}\NormalTok{)\{}
\NormalTok{    Ven.reset}\OperatorTok{$}\NormalTok{LO2.Set.Strata[i]=}\DecValTok{2}
\NormalTok{  \}  }\ControlFlowTok{else} \ControlFlowTok{if}\NormalTok{ (Ven.reset}\OperatorTok{$}\NormalTok{LO2.group2[i]}\OperatorTok{==}\DecValTok{1} \OperatorTok{&}\StringTok{ }\NormalTok{Ven.reset}\OperatorTok{$}\NormalTok{Setting[i] }\OperatorTok{==}\DecValTok{1}\NormalTok{)\{}
\NormalTok{    Ven.reset}\OperatorTok{$}\NormalTok{LO2.Set.Strata[i]=}\DecValTok{3}
\NormalTok{  \}  }\ControlFlowTok{else} \ControlFlowTok{if}\NormalTok{ (Ven.reset}\OperatorTok{$}\NormalTok{LO2.group2[i]}\OperatorTok{==}\DecValTok{2} \OperatorTok{&}\StringTok{ }\NormalTok{Ven.reset}\OperatorTok{$}\NormalTok{Setting[i] }\OperatorTok{==}\DecValTok{1}\NormalTok{)\{}
\NormalTok{    Ven.reset}\OperatorTok{$}\NormalTok{LO2.Set.Strata[i]=}\DecValTok{4}
\NormalTok{  \}  }\ControlFlowTok{else} \ControlFlowTok{if}\NormalTok{ (Ven.reset}\OperatorTok{$}\NormalTok{LO2.group2[i]}\OperatorTok{==}\DecValTok{1} \OperatorTok{&}\StringTok{ }\NormalTok{Ven.reset}\OperatorTok{$}\NormalTok{Setting[i] }\OperatorTok{==}\DecValTok{2}\NormalTok{)\{}
\NormalTok{    Ven.reset}\OperatorTok{$}\NormalTok{LO2.Set.Strata[i]=}\DecValTok{5}  
\NormalTok{  \}    }\ControlFlowTok{else} \ControlFlowTok{if}\NormalTok{ (Ven.reset}\OperatorTok{$}\NormalTok{LO2.group2[i]}\OperatorTok{==}\DecValTok{2} \OperatorTok{&}\StringTok{ }\NormalTok{Ven.reset}\OperatorTok{$}\NormalTok{Setting[i] }\OperatorTok{==}\DecValTok{2}\NormalTok{)\{}
\NormalTok{    Ven.reset}\OperatorTok{$}\NormalTok{LO2.Set.Strata[i]=}\DecValTok{6}\NormalTok{\}}
\NormalTok{\}}
\end{Highlighting}
\end{Shaded}

\begin{Shaded}
\begin{Highlighting}[]
\NormalTok{kmfitlST=}\KeywordTok{survfit}\NormalTok{(Y}\OperatorTok{~}\NormalTok{Ven.reset}\OperatorTok{$}\NormalTok{LO2.Set.Strata)}
\KeywordTok{plot}\NormalTok{(kmfitlST,}\DataTypeTok{fun=}\StringTok{"cloglog"}\NormalTok{,}\DataTypeTok{xlab=}\StringTok{"time in days on log scale"}\NormalTok{,}\DataTypeTok{ylab=}\StringTok{"log-log survival"}\NormalTok{, }\DataTypeTok{main=}\StringTok{"log-log curves by Setting and LO2 "}\NormalTok{,}\DataTypeTok{col=}\KeywordTok{c}\NormalTok{(}\StringTok{'red'}\NormalTok{,}\StringTok{'green'}\NormalTok{,}\StringTok{'blue'}\NormalTok{,}\StringTok{'cyan'}\NormalTok{,}\StringTok{'blueviolet'}\NormalTok{,}\StringTok{'brown'}\NormalTok{))}
\KeywordTok{legend}\NormalTok{(}\StringTok{"topleft"}\NormalTok{,}\DataTypeTok{cex=}\NormalTok{.}\DecValTok{7}\NormalTok{,}\KeywordTok{c}\NormalTok{(}\StringTok{"Set=0 and lo2=low"}\NormalTok{,}\StringTok{"Set=0 and lo2 =high"}\NormalTok{,}\StringTok{"Set=1 and lo2=low"}\NormalTok{,}\StringTok{"Set=1 and lo2=high"}\NormalTok{,}\StringTok{"Set=2 and lo2 = low"}\NormalTok{, }\StringTok{"Set=2 and lo2=hight"}\NormalTok{),}\DataTypeTok{lty=}\KeywordTok{c}\NormalTok{(}\StringTok{"solid"}\NormalTok{),}\DataTypeTok{col=}\KeywordTok{c}\NormalTok{(}\StringTok{'red'}\NormalTok{,}\StringTok{'green'}\NormalTok{,}\StringTok{'blue'}\NormalTok{,}\StringTok{'cyan'}\NormalTok{,}\StringTok{'blueviolet'}\NormalTok{,}\StringTok{'brown'}\NormalTok{))}
\end{Highlighting}
\end{Shaded}

\includegraphics{HW5_629Solutions_files/figure-latex/unnamed-chunk-9-1.pdf}
\textbf{Comment:}The curves stratified by LO2.group2 and Setting are
fairly parallel and straight. There are some deviations but there are 6
strata and the data is somewhat ``thinned out'' making the results less
reliable.

\hypertarget{fit-a-weibull-model-with-lo2-and-setting-in-the-model}{%
\subsection{3. Fit a Weibull model with LO2 and Setting in the
model}\label{fit-a-weibull-model-with-lo2-and-setting-in-the-model}}

\begin{Shaded}
\begin{Highlighting}[]
\NormalTok{mod.wbl1=}\KeywordTok{survreg}\NormalTok{(Y}\OperatorTok{~}\NormalTok{Setting}\OperatorTok{+}\NormalTok{LO2,}\DataTypeTok{data=}\NormalTok{Ven.reset,}\DataTypeTok{dist=}\StringTok{"weibull"}\NormalTok{)}
\end{Highlighting}
\end{Shaded}

\hypertarget{i.}{%
\subsection{3i.}\label{i.}}

\begin{Shaded}
\begin{Highlighting}[]
\KeywordTok{summary}\NormalTok{(mod.wbl1)}
\end{Highlighting}
\end{Shaded}

\begin{verbatim}
## 
## Call:
## survreg(formula = Y ~ Setting + LO2, data = Ven.reset, dist = "weibull")
##               Value Std. Error      z       p
## (Intercept)  2.6647     0.1032  25.81 < 2e-16
## Setting     -0.6501     0.0627 -10.37 < 2e-16
## LO2         -0.5903     0.0326 -18.09 < 2e-16
## Log(scale)  -0.4690     0.0641  -7.31 2.6e-13
## 
## Scale= 0.626 
## 
## Weibull distribution
## Loglik(model)= -229.6   Loglik(intercept only)= -335
##  Chisq= 210.85 on 2 degrees of freedom, p= 1.6e-46 
## Number of Newton-Raphson Iterations: 6 
## n= 150
\end{verbatim}

\textbf{Comment:} Both setting and LO2 are significant at
\(\alpha=0.05\) with p-values \(\approx0\).

\hypertarget{ii.-1}{%
\subsection{3ii.}\label{ii.-1}}

\begin{Shaded}
\begin{Highlighting}[]
\CommentTok{#Coefficients of Weibull model and 95% CI for AF for Setting }
\NormalTok{alpha0=}\KeywordTok{as.vector}\NormalTok{(mod.wbl1}\OperatorTok{$}\NormalTok{coefficients[}\DecValTok{1}\NormalTok{]) }
\NormalTok{alpha1=}\KeywordTok{as.vector}\NormalTok{(mod.wbl1}\OperatorTok{$}\NormalTok{coefficients[}\DecValTok{2}\NormalTok{])}
\NormalTok{alpha2=}\KeywordTok{as.vector}\NormalTok{(mod.wbl1}\OperatorTok{$}\NormalTok{coefficients[}\DecValTok{3}\NormalTok{])}
\CommentTok{#Approximate AF for Setting }
\NormalTok{AF=}\KeywordTok{exp}\NormalTok{(alpha1)}
\CommentTok{#CI for AF for Setting }
\NormalTok{lb.af =}\StringTok{ }\KeywordTok{exp}\NormalTok{(alpha1 }\OperatorTok{-}\StringTok{ }\FloatTok{1.96}\OperatorTok{*}\FloatTok{0.0627}\NormalTok{)}
\NormalTok{ub.af =}\StringTok{ }\KeywordTok{exp}\NormalTok{(alpha1 }\OperatorTok{+}\StringTok{ }\FloatTok{1.96}\OperatorTok{*}\FloatTok{0.0627}\NormalTok{)}
\KeywordTok{print}\NormalTok{(}\KeywordTok{paste}\NormalTok{(}\StringTok{"the estimated AF for setting is:"}\NormalTok{,}\KeywordTok{round}\NormalTok{(AF,}\DecValTok{4}\NormalTok{)))}
\end{Highlighting}
\end{Shaded}

\begin{verbatim}
## [1] "the estimated AF for setting is: 0.522"
\end{verbatim}

\begin{Shaded}
\begin{Highlighting}[]
\KeywordTok{print}\NormalTok{(}\KeywordTok{paste}\NormalTok{(}\StringTok{"the lower bound for the 95% CI for the AF of setting is:"}\NormalTok{,}\KeywordTok{round}\NormalTok{(lb.af,}\DecValTok{4}\NormalTok{)))}
\end{Highlighting}
\end{Shaded}

\begin{verbatim}
## [1] "the lower bound for the 95% CI for the AF of setting is: 0.4616"
\end{verbatim}

\begin{Shaded}
\begin{Highlighting}[]
\KeywordTok{print}\NormalTok{(}\KeywordTok{paste}\NormalTok{(}\StringTok{"the upper bound for the 95% CI for the AF of setting is:"}\NormalTok{,}\KeywordTok{round}\NormalTok{(ub.af,}\DecValTok{4}\NormalTok{)))}
\end{Highlighting}
\end{Shaded}

\begin{verbatim}
## [1] "the upper bound for the 95% CI for the AF of setting is: 0.5903"
\end{verbatim}

\hypertarget{iii.}{%
\subsection{3iii.}\label{iii.}}

\begin{Shaded}
\begin{Highlighting}[]
\CommentTok{#Shape parameter }
\NormalTok{p=}\DecValTok{1}\OperatorTok{/}\NormalTok{mod.wbl1}\OperatorTok{$}\NormalTok{scale}
\CommentTok{#Coefficients of PH model and 95% CI for HR for Setting }
\NormalTok{beta0=}\OperatorTok{-}\NormalTok{p}\OperatorTok{*}\NormalTok{alpha0; beta1 =}\StringTok{ }\OperatorTok{-}\NormalTok{p}\OperatorTok{*}\NormalTok{alpha1; beta2=}\StringTok{ }\OperatorTok{-}\NormalTok{p}\OperatorTok{*}\NormalTok{alpha2}
\CommentTok{#Approximate Hazard Ratio for Setting }
\NormalTok{HR=}\KeywordTok{exp}\NormalTok{(beta1);}
\CommentTok{#Confidence Interval for HR for Setting }
\NormalTok{lb.hr =}\StringTok{ }\DecValTok{1}\OperatorTok{/}\NormalTok{(ub.af)}\OperatorTok{^}\NormalTok{p}
\NormalTok{ub.hr =}\StringTok{ }\DecValTok{1}\OperatorTok{/}\NormalTok{(lb.af)}\OperatorTok{^}\NormalTok{p}
\KeywordTok{print}\NormalTok{(}\KeywordTok{paste}\NormalTok{(}\StringTok{"the estimated HR for setting is:"}\NormalTok{,}\KeywordTok{round}\NormalTok{(HR,}\DecValTok{4}\NormalTok{)))}
\end{Highlighting}
\end{Shaded}

\begin{verbatim}
## [1] "the estimated HR for setting is: 2.8266"
\end{verbatim}

\begin{Shaded}
\begin{Highlighting}[]
\KeywordTok{print}\NormalTok{(}\KeywordTok{paste}\NormalTok{(}\StringTok{"the lower bound for the 95% CI for the HR of setting is:"}\NormalTok{,}\KeywordTok{round}\NormalTok{(lb.hr,}\DecValTok{4}\NormalTok{)))}
\end{Highlighting}
\end{Shaded}

\begin{verbatim}
## [1] "the lower bound for the 95% CI for the HR of setting is: 2.3225"
\end{verbatim}

\begin{Shaded}
\begin{Highlighting}[]
\KeywordTok{print}\NormalTok{(}\KeywordTok{paste}\NormalTok{(}\StringTok{"the upper bound for the 95% CI for the HR of setting is:"}\NormalTok{,}\KeywordTok{round}\NormalTok{(ub.hr,}\DecValTok{4}\NormalTok{)))}
\end{Highlighting}
\end{Shaded}

\begin{verbatim}
## [1] "the upper bound for the 95% CI for the HR of setting is: 3.4401"
\end{verbatim}

\hypertarget{fit-a-weibull-model-with-lo2-and-setting-in-the-model-and-interaction-terms-between-lo2-and-setting}{%
\subsection{4. Fit a Weibull model with LO2 and Setting in the model and
interaction terms between LO2 and
Setting}\label{fit-a-weibull-model-with-lo2-and-setting-in-the-model-and-interaction-terms-between-lo2-and-setting}}

\begin{Shaded}
\begin{Highlighting}[]
\NormalTok{mod.wbl1.int=}\KeywordTok{survreg}\NormalTok{(Y}\OperatorTok{~}\NormalTok{Setting}\OperatorTok{+}\NormalTok{LO2}\OperatorTok{+}\NormalTok{Setting}\OperatorTok{:}\NormalTok{LO2,}\DataTypeTok{data=}\NormalTok{Ven.reset,}\DataTypeTok{dist=}\StringTok{"weibull"}\NormalTok{)}
\NormalTok{alpha0=}\KeywordTok{as.vector}\NormalTok{(mod.wbl1.int}\OperatorTok{$}\NormalTok{coefficients[}\DecValTok{1}\NormalTok{]) }
\NormalTok{alpha1=}\KeywordTok{as.vector}\NormalTok{(mod.wbl1.int}\OperatorTok{$}\NormalTok{coefficients[}\DecValTok{2}\NormalTok{])}
\NormalTok{alpha2=}\KeywordTok{as.vector}\NormalTok{(mod.wbl1.int}\OperatorTok{$}\NormalTok{coefficients[}\DecValTok{3}\NormalTok{])}
\NormalTok{alpha3=}\KeywordTok{as.vector}\NormalTok{(mod.wbl1.int}\OperatorTok{$}\NormalTok{coefficients[}\DecValTok{4}\NormalTok{])}
\end{Highlighting}
\end{Shaded}

\hypertarget{i.-1}{%
\subsection{4i.}\label{i.-1}}

\begin{Shaded}
\begin{Highlighting}[]
\KeywordTok{summary}\NormalTok{(mod.wbl1.int)}
\end{Highlighting}
\end{Shaded}

\begin{verbatim}
## 
## Call:
## survreg(formula = Y ~ Setting + LO2 + Setting:LO2, data = Ven.reset, 
##     dist = "weibull")
##               Value Std. Error      z       p
## (Intercept)  2.8599     0.1455  19.66 < 2e-16
## Setting     -0.8136     0.1006  -8.09 6.1e-16
## LO2         -0.6864     0.0563 -12.19 < 2e-16
## Setting:LO2  0.0841     0.0401   2.10   0.036
## Log(scale)  -0.4832     0.0646  -7.48 7.2e-14
## 
## Scale= 0.617 
## 
## Weibull distribution
## Loglik(model)= -227.4   Loglik(intercept only)= -335
##  Chisq= 215.12 on 3 degrees of freedom, p= 2.3e-46 
## Number of Newton-Raphson Iterations: 6 
## n= 150
\end{verbatim}

\begin{Shaded}
\begin{Highlighting}[]
\CommentTok{#Likelihood ratio test for interaction }
\NormalTok{CSstat =}\StringTok{ }\DecValTok{-2}\OperatorTok{*}\NormalTok{(mod.wbl1}\OperatorTok{$}\NormalTok{loglik[}\DecValTok{2}\NormalTok{]}\OperatorTok{-}\NormalTok{mod.wbl1.int}\OperatorTok{$}\NormalTok{loglik[}\DecValTok{2}\NormalTok{])}
\KeywordTok{print}\NormalTok{(}\KeywordTok{paste}\NormalTok{(}\StringTok{"The value of our test statistic is"}\NormalTok{, }\KeywordTok{round}\NormalTok{(CSstat,}\DecValTok{2}\NormalTok{)))}
\end{Highlighting}
\end{Shaded}

\begin{verbatim}
## [1] "The value of our test statistic is 4.26"
\end{verbatim}

\begin{Shaded}
\begin{Highlighting}[]
\NormalTok{CV =}\StringTok{ }\KeywordTok{qchisq}\NormalTok{(.}\DecValTok{95}\NormalTok{,}\DataTypeTok{df=}\DecValTok{1}\NormalTok{)}
\KeywordTok{print}\NormalTok{(}\KeywordTok{paste}\NormalTok{(}\StringTok{"Our critical value is:"}\NormalTok{,}\KeywordTok{round}\NormalTok{(CV,}\DecValTok{4}\NormalTok{)))}
\end{Highlighting}
\end{Shaded}

\begin{verbatim}
## [1] "Our critical value is: 3.8415"
\end{verbatim}

\begin{Shaded}
\begin{Highlighting}[]
\NormalTok{CSstat}\OperatorTok{>}\NormalTok{CV}
\end{Highlighting}
\end{Shaded}

\begin{verbatim}
## [1] TRUE
\end{verbatim}

\begin{Shaded}
\begin{Highlighting}[]
\NormalTok{pvalue1=}\KeywordTok{pchisq}\NormalTok{(CSstat,}\DataTypeTok{df =} \DecValTok{1}\NormalTok{,}\DataTypeTok{lower.tail =} \OtherTok{FALSE}\NormalTok{)}
\KeywordTok{print}\NormalTok{(}\KeywordTok{paste}\NormalTok{(}\StringTok{"Our p value is:"}\NormalTok{,}\KeywordTok{round}\NormalTok{(pvalue1,}\DecValTok{4}\NormalTok{)))}
\end{Highlighting}
\end{Shaded}

\begin{verbatim}
## [1] "Our p value is: 0.039"
\end{verbatim}

\textbf{Comment:} The p-value for our Wald test is 0.036 and the p-value
for our likelihood ratio test is 0.039. In both cases, we reject
\(H_0: \alpha_3=0\) at significance level \(\alpha=0.05\) and
interaction is significant.

\hypertarget{ii.-2}{%
\subsection{4ii.}\label{ii.-2}}

\begin{Shaded}
\begin{Highlighting}[]
\CommentTok{#shape parameter}
\NormalTok{p=}\DecValTok{1}\OperatorTok{/}\NormalTok{mod.wbl1.int}\OperatorTok{$}\NormalTok{scale}
\NormalTok{Q1 =}\StringTok{ }\KeywordTok{as.vector}\NormalTok{(}\KeywordTok{quantile}\NormalTok{(Ven.reset}\OperatorTok{$}\NormalTok{LO2)[}\DecValTok{2}\NormalTok{]); Q2 =}\StringTok{ }\KeywordTok{as.vector}\NormalTok{(}\KeywordTok{quantile}\NormalTok{(Ven.reset}\OperatorTok{$}\NormalTok{LO2)[}\DecValTok{3}\NormalTok{]) }
\NormalTok{Q3 =}\StringTok{ }\KeywordTok{as.vector}\NormalTok{(}\KeywordTok{quantile}\NormalTok{(Ven.reset}\OperatorTok{$}\NormalTok{LO2)[}\DecValTok{4}\NormalTok{])}
\NormalTok{AF1 =}\StringTok{ }\KeywordTok{exp}\NormalTok{(alpha1 }\OperatorTok{+}\StringTok{ }\NormalTok{Q1}\OperatorTok{*}\NormalTok{alpha3); AF2 =}\StringTok{ }\KeywordTok{exp}\NormalTok{(alpha1 }\OperatorTok{+}\StringTok{ }\NormalTok{Q2}\OperatorTok{*}\NormalTok{alpha3); AF3 =}\StringTok{ }\KeywordTok{exp}\NormalTok{(alpha1 }\OperatorTok{+}\StringTok{ }\NormalTok{Q3}\OperatorTok{*}\NormalTok{alpha3)}
\NormalTok{HR1 =}\StringTok{ }\KeywordTok{exp}\NormalTok{(}\OperatorTok{-}\NormalTok{p}\OperatorTok{*}\NormalTok{alpha1 }\OperatorTok{-}\StringTok{ }\NormalTok{Q1}\OperatorTok{*}\NormalTok{p}\OperatorTok{*}\NormalTok{alpha3); HR2 =}\StringTok{ }\KeywordTok{exp}\NormalTok{(}\OperatorTok{-}\NormalTok{p}\OperatorTok{*}\NormalTok{alpha1 }\OperatorTok{-}\StringTok{ }\NormalTok{p}\OperatorTok{*}\NormalTok{Q2}\OperatorTok{*}\NormalTok{alpha3); HR3 =}\StringTok{ }\KeywordTok{exp}\NormalTok{(}\OperatorTok{-}\NormalTok{alpha1 }\OperatorTok{-}\StringTok{ }\NormalTok{p}\OperatorTok{*}\NormalTok{Q3}\OperatorTok{*}\NormalTok{alpha3)}
\end{Highlighting}
\end{Shaded}

\textbf{Comment:} The AFs for Setting at LO2 Q1, Q2 and Q3,
respectively, are: 0.4824747, 0.5260169 and 0.5744542. The HRs for
Setting at LO2 Q1, Q2, and Q3, respectively, 3.259815, and 2.833689 and
1.481828.

\hypertarget{iii.-1}{%
\subsection{4iii.}\label{iii.-1}}

\begin{Shaded}
\begin{Highlighting}[]
\CommentTok{#For Setting=0}
\NormalTok{LO2mean=}\KeywordTok{mean}\NormalTok{(Ven.reset}\OperatorTok{$}\NormalTok{LO2)}
\NormalTok{pattern1=}\KeywordTok{data.frame}\NormalTok{(}\DataTypeTok{Setting=}\DecValTok{0}\NormalTok{,}\DataTypeTok{LO2=}\NormalTok{LO2mean)}
\NormalTok{pct2=}\DecValTok{0}\OperatorTok{:}\DecValTok{1000}\OperatorTok{/}\DecValTok{1000}
\NormalTok{days1=}\KeywordTok{predict}\NormalTok{(mod.wbl1.int,}\DataTypeTok{newdata=}\NormalTok{pattern1,}
              \DataTypeTok{type=}\StringTok{"quantile"}\NormalTok{,}\DataTypeTok{p=}\NormalTok{pct2)}
\NormalTok{survival=}\DecValTok{1}\OperatorTok{-}\NormalTok{pct2}
\KeywordTok{plot}\NormalTok{(days1,survival,}\DataTypeTok{xlab=}\StringTok{"survival time in days"}\NormalTok{,}\DataTypeTok{ylab=} \StringTok{"survival}
\StringTok{     probabilities"}\NormalTok{,}\DataTypeTok{xlim=}\KeywordTok{c}\NormalTok{(}\DecValTok{0}\NormalTok{,}\DecValTok{15}\NormalTok{),}\DataTypeTok{col=}\KeywordTok{c}\NormalTok{(}\StringTok{'blue'}\NormalTok{))}
\KeywordTok{par}\NormalTok{(}\DataTypeTok{new=}\OtherTok{TRUE}\NormalTok{)}
\CommentTok{#For Setting=1}
\NormalTok{pattern2=}\KeywordTok{data.frame}\NormalTok{(}\DataTypeTok{Setting=}\DecValTok{1}\NormalTok{,}\DataTypeTok{LO2=}\NormalTok{LO2mean)}
\NormalTok{pct2=}\DecValTok{0}\OperatorTok{:}\DecValTok{1000}\OperatorTok{/}\DecValTok{1000}
\NormalTok{days2=}\KeywordTok{predict}\NormalTok{(mod.wbl1.int,}\DataTypeTok{newdata=}\NormalTok{pattern2,}
              \DataTypeTok{type=}\StringTok{"quantile"}\NormalTok{,}\DataTypeTok{p=}\NormalTok{pct2)}
\NormalTok{survival=}\DecValTok{1}\OperatorTok{-}\NormalTok{pct2}
\KeywordTok{plot}\NormalTok{(days2,survival,}\DataTypeTok{xlab=}\StringTok{"survival time in days"}\NormalTok{,}\DataTypeTok{ylab=} \StringTok{"survival}
\StringTok{     probabilities"}\NormalTok{,}\DataTypeTok{col=}\KeywordTok{c}\NormalTok{(}\StringTok{'red'}\NormalTok{),}\DataTypeTok{xlim=}\KeywordTok{c}\NormalTok{(}\DecValTok{0}\NormalTok{,}\DecValTok{15}\NormalTok{))}
\KeywordTok{par}\NormalTok{(}\DataTypeTok{new=}\OtherTok{TRUE}\NormalTok{)}
\CommentTok{#For Setting=2}
\NormalTok{pattern2=}\KeywordTok{data.frame}\NormalTok{(}\DataTypeTok{Setting=}\DecValTok{2}\NormalTok{,}\DataTypeTok{LO2=}\NormalTok{LO2mean)}
\NormalTok{pct2=}\DecValTok{0}\OperatorTok{:}\DecValTok{1000}\OperatorTok{/}\DecValTok{1000}
\NormalTok{days3=}\KeywordTok{predict}\NormalTok{(mod.wbl1.int,}\DataTypeTok{newdata=}\NormalTok{pattern2,}
              \DataTypeTok{type=}\StringTok{"quantile"}\NormalTok{,}\DataTypeTok{p=}\NormalTok{pct2)}
\NormalTok{survival=}\DecValTok{1}\OperatorTok{-}\NormalTok{pct2}
\KeywordTok{plot}\NormalTok{(days3,survival,}\DataTypeTok{xlab=}\StringTok{"survival time in days"}\NormalTok{,}\DataTypeTok{ylab=} \StringTok{"survival}
\StringTok{     probabilities"}\NormalTok{,}\DataTypeTok{col=}\KeywordTok{c}\NormalTok{(}\StringTok{'green'}\NormalTok{),}\DataTypeTok{xlim=}\KeywordTok{c}\NormalTok{(}\DecValTok{0}\NormalTok{,}\DecValTok{15}\NormalTok{))}
\KeywordTok{legend}\NormalTok{(}\StringTok{"topright"}\NormalTok{,}\KeywordTok{c}\NormalTok{(}\StringTok{"Setting=0"}\NormalTok{, }\StringTok{"Setting=1"}\NormalTok{, }\StringTok{"Setting=2"}\NormalTok{), }
\DataTypeTok{col=}\KeywordTok{c}\NormalTok{(}\StringTok{'blue'}\NormalTok{,}\StringTok{'red'}\NormalTok{,}\StringTok{'green'}\NormalTok{), }\DataTypeTok{lty=}\KeywordTok{c}\NormalTok{(}\StringTok{"solid"}\NormalTok{)) }
\end{Highlighting}
\end{Shaded}

\includegraphics{HW5_629Solutions_files/figure-latex/unnamed-chunk-17-1.pdf}

\hypertarget{iv.}{%
\subsection{4iv.}\label{iv.}}

\textbf{Comment:} As setting increases (from 0 to 1 to 2) the survival
experience ``accelerates'' and the ventilator is more likely to need to
be reset sooner. This effect decreases with higher LO2.

\hypertarget{i.-2}{%
\subsection{5i.}\label{i.-2}}

\begin{Shaded}
\begin{Highlighting}[]
\KeywordTok{plot}\NormalTok{(kmfitO24,}\DataTypeTok{xlab=}\StringTok{"time in days on log scale"}\NormalTok{,}\DataTypeTok{ylab=}\StringTok{"log-log survival"}\NormalTok{, }\DataTypeTok{main=}\StringTok{"KM curves by LO2group"}\NormalTok{,}\DataTypeTok{col=}\KeywordTok{c}\NormalTok{(}\StringTok{"red"}\NormalTok{,}\StringTok{"green"}\NormalTok{,}\StringTok{"blue"}\NormalTok{,}\StringTok{"black"}\NormalTok{))}
\KeywordTok{legend}\NormalTok{(}\StringTok{"topright"}\NormalTok{,}\DataTypeTok{cex=}\NormalTok{.}\DecValTok{9}\NormalTok{,}\KeywordTok{c}\NormalTok{(}\StringTok{"LO2=low"}\NormalTok{,}\StringTok{"LO2=medlow"}\NormalTok{,}\StringTok{"LO2=medhigh"}\NormalTok{,}\StringTok{"LO2=high"}\NormalTok{),}\DataTypeTok{lty=}\KeywordTok{c}\NormalTok{(}\StringTok{"solid"}\NormalTok{),}\DataTypeTok{col=}\KeywordTok{c}\NormalTok{(}\StringTok{"red"}\NormalTok{,}\StringTok{"green"}\NormalTok{,}\StringTok{"blue"}\NormalTok{,}\StringTok{"black"}\NormalTok{))}
\end{Highlighting}
\end{Shaded}

\includegraphics{HW5_629Solutions_files/figure-latex/unnamed-chunk-18-1.pdf}

\begin{Shaded}
\begin{Highlighting}[]
\CommentTok{#log-rank test}
\KeywordTok{survdiff}\NormalTok{(Y}\OperatorTok{~}\NormalTok{Ven.reset}\OperatorTok{$}\NormalTok{LO2.group)}
\end{Highlighting}
\end{Shaded}

\begin{verbatim}
## Call:
## survdiff(formula = Y ~ Ven.reset$LO2.group)
## 
##                        N Observed Expected (O-E)^2/E (O-E)^2/V
## Ven.reset$LO2.group=1 38       33     71.8   20.9346    48.128
## Ven.reset$LO2.group=2 37       37     38.9    0.0962     0.133
## Ven.reset$LO2.group=3 37       37     22.6    9.2358    11.674
## Ven.reset$LO2.group=4 38       38     11.7   58.7228    69.692
## 
##  Chisq= 108  on 3 degrees of freedom, p= <2e-16
\end{verbatim}

\textbf{Comment:} With p-value \(\approx0\) we reject \(H_0:\) that the
survival experience of the four LO2 groups is the same at significance
level \(\alpha=0.05\).

\hypertarget{ii.-3}{%
\subsection{5ii.}\label{ii.-3}}

\begin{Shaded}
\begin{Highlighting}[]
\KeywordTok{plot}\NormalTok{(}\KeywordTok{log}\NormalTok{(kmfitST3}\OperatorTok{$}\NormalTok{time),}\KeywordTok{log}\NormalTok{(kmfitST3}\OperatorTok{$}\NormalTok{surv}\OperatorTok{/}\NormalTok{(}\DecValTok{1}\OperatorTok{-}\NormalTok{kmfitST3}\OperatorTok{$}\NormalTok{surv)),}\DataTypeTok{xlab=}\StringTok{"ln(time)"}\NormalTok{,}\DataTypeTok{ylab=}\StringTok{"ln[(1-S(t))/S(t))]"}\NormalTok{,}\DataTypeTok{main=}\StringTok{"ln(t) vs Failure Odds by Setting"}\NormalTok{)}
\end{Highlighting}
\end{Shaded}

\includegraphics{HW5_629Solutions_files/figure-latex/unnamed-chunk-20-1.pdf}
\textbf{Comment:} The plot of ln(t) against estimated failure odds for
Setting don't look approximately parallel and straight suggesting a
log-logistic model for LO2.group alone is not appropriate.

\hypertarget{iii.-2}{%
\subsection{5iii.}\label{iii.-2}}

\begin{Shaded}
\begin{Highlighting}[]
\KeywordTok{plot}\NormalTok{(}\KeywordTok{log}\NormalTok{(kmfitO24}\OperatorTok{$}\NormalTok{time),}\KeywordTok{log}\NormalTok{(kmfitO24}\OperatorTok{$}\NormalTok{surv}\OperatorTok{/}\NormalTok{(}\DecValTok{1}\OperatorTok{-}\NormalTok{kmfitO24}\OperatorTok{$}\NormalTok{surv)),}\DataTypeTok{xlab=}\StringTok{"ln(time)"}\NormalTok{,}\DataTypeTok{ylab=}\StringTok{"ln[(1-S(t))/S(t))]"}\NormalTok{,}\DataTypeTok{main=}\StringTok{"ln(t) vs Failure Odds by LO2.group"}\NormalTok{)}
\end{Highlighting}
\end{Shaded}

\includegraphics{HW5_629Solutions_files/figure-latex/unnamed-chunk-21-1.pdf}
\textbf{Comment:} The plot of ln(t) against estimated failure odds for
LO2.group look approximately parallel and straight suggesting a
log-logistic model for LO2.group alone is appropriate.

\hypertarget{iv.-1}{%
\subsection{5iv.}\label{iv.-1}}

\begin{Shaded}
\begin{Highlighting}[]
\NormalTok{mod.LO2.lgl=}\KeywordTok{survreg}\NormalTok{(Y }\OperatorTok{~}\StringTok{ }\NormalTok{LO2,}\DataTypeTok{data=}\NormalTok{Ven.reset,}\DataTypeTok{dist=}\StringTok{"loglogistic"}\NormalTok{)}
\KeywordTok{summary}\NormalTok{(mod.LO2.lgl)}
\end{Highlighting}
\end{Shaded}

\begin{verbatim}
## 
## Call:
## survreg(formula = Y ~ LO2, data = Ven.reset, dist = "loglogistic")
##               Value Std. Error      z      p
## (Intercept)  1.7193     0.1154  14.90 <2e-16
## LO2         -0.6044     0.0452 -13.37 <2e-16
## Log(scale)  -0.6046     0.0682  -8.87 <2e-16
## 
## Scale= 0.546 
## 
## Log logistic distribution
## Loglik(model)= -270.2   Loglik(intercept only)= -331
##  Chisq= 121.62 on 1 degrees of freedom, p= 2.8e-28 
## Number of Newton-Raphson Iterations: 5 
## n= 150
\end{verbatim}

\begin{Shaded}
\begin{Highlighting}[]
\NormalTok{alpha0=}\KeywordTok{as.vector}\NormalTok{(mod.LO2.lgl}\OperatorTok{$}\NormalTok{coefficients[}\DecValTok{1}\NormalTok{]); alpha1=}\KeywordTok{as.vector}\NormalTok{(mod.LO2.lgl}\OperatorTok{$}\NormalTok{coefficients[}\DecValTok{2}\NormalTok{]);}
\NormalTok{AF=}\KeywordTok{exp}\NormalTok{(alpha1)}
\CommentTok{#95% CI for AF }
\NormalTok{lb.af =}\StringTok{ }\KeywordTok{exp}\NormalTok{(alpha1 }\OperatorTok{-}\StringTok{ }\FloatTok{1.96}\OperatorTok{*}\FloatTok{0.0425}\NormalTok{)}
\NormalTok{ub.af =}\StringTok{ }\KeywordTok{exp}\NormalTok{(alpha1 }\OperatorTok{+}\StringTok{ }\FloatTok{1.96}\OperatorTok{*}\FloatTok{0.0425}\NormalTok{)}
\KeywordTok{print}\NormalTok{(}\KeywordTok{paste}\NormalTok{(}\StringTok{"The approximate AF for LO2 is: "}\NormalTok{,AF))}
\end{Highlighting}
\end{Shaded}

\begin{verbatim}
## [1] "The approximate AF for LO2 is:  0.546428183734623"
\end{verbatim}

\begin{Shaded}
\begin{Highlighting}[]
\KeywordTok{print}\NormalTok{(}\KeywordTok{paste}\NormalTok{(}\StringTok{"The lower bound for the AF for LO2 is: "}\NormalTok{,lb.af))}
\end{Highlighting}
\end{Shaded}

\begin{verbatim}
## [1] "The lower bound for the AF for LO2 is:  0.502754956660336"
\end{verbatim}

\begin{Shaded}
\begin{Highlighting}[]
\KeywordTok{print}\NormalTok{(}\KeywordTok{paste}\NormalTok{(}\StringTok{"The upper bound for the AF for LO2 is: "}\NormalTok{,ub.af))}
\end{Highlighting}
\end{Shaded}

\begin{verbatim}
## [1] "The upper bound for the AF for LO2 is:  0.593895208836785"
\end{verbatim}

\textbf{Comment:} L02 is significant by Wald test with p-value
\(\approx 0\).

\hypertarget{v.}{%
\subsection{5v.}\label{v.}}

\begin{Shaded}
\begin{Highlighting}[]
\CommentTok{#shape parameter}
\NormalTok{p=}\DecValTok{1}\OperatorTok{/}\NormalTok{mod.LO2.lgl}\OperatorTok{$}\NormalTok{scale}
\CommentTok{#PO Model }
\NormalTok{beta0=}\OperatorTok{-}\NormalTok{p}\OperatorTok{*}\NormalTok{alpha0; beta1 =}\StringTok{ }\OperatorTok{-}\NormalTok{p}\OperatorTok{*}\NormalTok{alpha1}
\NormalTok{FOR=}\KeywordTok{exp}\NormalTok{(beta1);}
\CommentTok{#95% CI for FOR}
\NormalTok{lb.for =}\StringTok{ }\NormalTok{(}\DecValTok{1}\OperatorTok{/}\NormalTok{ub.af)}\OperatorTok{^}\NormalTok{p }
\NormalTok{ub.for =}\StringTok{ }\NormalTok{(}\DecValTok{1}\OperatorTok{/}\NormalTok{lb.af)}\OperatorTok{^}\NormalTok{p}
\KeywordTok{print}\NormalTok{(}\KeywordTok{paste}\NormalTok{(}\StringTok{"The approximate FOR for LO2 is: "}\NormalTok{,}\KeywordTok{round}\NormalTok{(FOR,}\DecValTok{4}\NormalTok{)))}
\end{Highlighting}
\end{Shaded}

\begin{verbatim}
## [1] "The approximate FOR for LO2 is:  3.023"
\end{verbatim}

\begin{Shaded}
\begin{Highlighting}[]
\KeywordTok{print}\NormalTok{(}\KeywordTok{paste}\NormalTok{(}\StringTok{"The lower bound for the AF for LO2 is: "}\NormalTok{,}\KeywordTok{round}\NormalTok{(lb.for,}\DecValTok{4}\NormalTok{)))}
\end{Highlighting}
\end{Shaded}

\begin{verbatim}
## [1] "The lower bound for the AF for LO2 is:  2.5955"
\end{verbatim}

\begin{Shaded}
\begin{Highlighting}[]
\KeywordTok{print}\NormalTok{(}\KeywordTok{paste}\NormalTok{(}\StringTok{"The upper bound for the AF for LO2 is: "}\NormalTok{,}\KeywordTok{round}\NormalTok{(ub.for,}\DecValTok{4}\NormalTok{)))}
\end{Highlighting}
\end{Shaded}

\begin{verbatim}
## [1] "The upper bound for the AF for LO2 is:  3.5209"
\end{verbatim}

\hypertarget{vi.}{%
\subsection{5vi.}\label{vi.}}

\begin{Shaded}
\begin{Highlighting}[]
\CommentTok{#find means by LO2.group }
\NormalTok{Ven.reset }\OperatorTok
\StringTok{  }\KeywordTok{group_by}\NormalTok{(LO2.group) }\OperatorTok
\StringTok{  }\KeywordTok{summarise}\NormalTok{(}\DataTypeTok{mean_iron =} \KeywordTok{mean}\NormalTok{(LO2),}
            \DataTypeTok{number_obs =} \KeywordTok{n}\NormalTok{())}
\end{Highlighting}
\end{Shaded}

\begin{verbatim}
## # A tibble: 4 x 3
##   LO2.group mean_iron number_obs
##   <fct>         <dbl>      <int>
## 1 1            -0.421         38
## 2 2             1.45          37
## 3 3             2.44          37
## 4 4             4.01          38
\end{verbatim}

\begin{Shaded}
\begin{Highlighting}[]
\CommentTok{#for LO2=-0.421}
\NormalTok{pattern1=}\KeywordTok{data.frame}\NormalTok{(}\DataTypeTok{LO2=}\OperatorTok{-}\FloatTok{0.421}\NormalTok{)}
\NormalTok{pct2=}\DecValTok{0}\OperatorTok{:}\DecValTok{1000}\OperatorTok{/}\DecValTok{1000}
\NormalTok{days1=}\KeywordTok{predict}\NormalTok{(mod.LO2.lgl,}\DataTypeTok{newdata=}\NormalTok{pattern1,}
              \DataTypeTok{type=}\StringTok{"quantile"}\NormalTok{,}\DataTypeTok{p=}\NormalTok{pct2)}
\NormalTok{survival=}\DecValTok{1}\OperatorTok{-}\NormalTok{pct2}
\KeywordTok{plot}\NormalTok{(days1,survival,}\DataTypeTok{xlab=}\StringTok{"survival time in days"}\NormalTok{,}\DataTypeTok{ylab=}\StringTok{"survival}
\StringTok{     probabilities"}\NormalTok{,}\DataTypeTok{main=}\StringTok{"log-logistic survival estimates for LO2=-0.421,1.45, 2.44,4.01"}\NormalTok{,}\DataTypeTok{xlim=}\KeywordTok{c}\NormalTok{(}\DecValTok{0}\NormalTok{,}\DecValTok{25}\NormalTok{),}\DataTypeTok{col=}\KeywordTok{c}\NormalTok{(}\StringTok{'blue'}\NormalTok{))}
\KeywordTok{par}\NormalTok{(}\DataTypeTok{new=}\OtherTok{TRUE}\NormalTok{)}
\CommentTok{#for LO2=1.45}
\NormalTok{pattern1=}\KeywordTok{data.frame}\NormalTok{(}\DataTypeTok{LO2=}\FloatTok{1.45}\NormalTok{)}
\NormalTok{pct2=}\DecValTok{0}\OperatorTok{:}\DecValTok{1000}\OperatorTok{/}\DecValTok{1000}
\NormalTok{days2=}\KeywordTok{predict}\NormalTok{(mod.LO2.lgl,}\DataTypeTok{newdata=}\NormalTok{pattern1,}
              \DataTypeTok{type=}\StringTok{"quantile"}\NormalTok{,}\DataTypeTok{p=}\NormalTok{pct2)}
\NormalTok{survival=}\DecValTok{1}\OperatorTok{-}\NormalTok{pct2}
\KeywordTok{plot}\NormalTok{(days2,survival,}\DataTypeTok{xlim=}\KeywordTok{c}\NormalTok{(}\DecValTok{0}\NormalTok{,}\DecValTok{25}\NormalTok{),}\DataTypeTok{xlab=}\StringTok{"survival time in days"}\NormalTok{,}\DataTypeTok{ylab=} \StringTok{"survival probabilities"}\NormalTok{,}\DataTypeTok{col=}\KeywordTok{c}\NormalTok{(}\StringTok{'red'}\NormalTok{))}
\KeywordTok{par}\NormalTok{(}\DataTypeTok{new=}\OtherTok{TRUE}\NormalTok{)}
\CommentTok{#for LO2=2.44}
\NormalTok{pattern3=}\KeywordTok{data.frame}\NormalTok{(}\DataTypeTok{LO2=}\FloatTok{2.44}\NormalTok{)}
\NormalTok{pct2=}\DecValTok{0}\OperatorTok{:}\DecValTok{1000}\OperatorTok{/}\DecValTok{1000}
\NormalTok{days3=}\KeywordTok{predict}\NormalTok{(mod.LO2.lgl,}\DataTypeTok{newdata=}\NormalTok{pattern3,}
              \DataTypeTok{type=}\StringTok{"quantile"}\NormalTok{,}\DataTypeTok{p=}\NormalTok{pct2)}
\KeywordTok{plot}\NormalTok{(days3,survival,}\DataTypeTok{xlab=}\StringTok{"survival time in days"}\NormalTok{,}\DataTypeTok{ylab=}\StringTok{"survival}
\StringTok{     probabilities"}\NormalTok{,}\DataTypeTok{col=}\KeywordTok{c}\NormalTok{(}\StringTok{'green'}\NormalTok{),}\DataTypeTok{xlim=}\KeywordTok{c}\NormalTok{(}\DecValTok{0}\NormalTok{,}\DecValTok{25}\NormalTok{))}
\KeywordTok{par}\NormalTok{(}\DataTypeTok{new=}\OtherTok{TRUE}\NormalTok{)}
\CommentTok{#for LO2=4.01}
\NormalTok{pattern3=}\KeywordTok{data.frame}\NormalTok{(}\DataTypeTok{LO2=}\FloatTok{4.01}\NormalTok{)}
\NormalTok{pct2=}\DecValTok{0}\OperatorTok{:}\DecValTok{1000}\OperatorTok{/}\DecValTok{1000}
\NormalTok{days4=}\KeywordTok{predict}\NormalTok{(mod.LO2.lgl,}\DataTypeTok{newdata=}\NormalTok{pattern3,}
              \DataTypeTok{type=}\StringTok{"quantile"}\NormalTok{,}\DataTypeTok{p=}\NormalTok{pct2)}
\KeywordTok{plot}\NormalTok{(days4,survival,}\DataTypeTok{xlab=}\StringTok{"survival time in days"}\NormalTok{,}\DataTypeTok{ylab=}\StringTok{"survival}
\StringTok{     probabilities"}\NormalTok{,}\DataTypeTok{xlim=}\KeywordTok{c}\NormalTok{(}\DecValTok{0}\NormalTok{,}\DecValTok{25}\NormalTok{),}\DataTypeTok{col=}\KeywordTok{c}\NormalTok{(}\StringTok{'cyan'}\NormalTok{))}
\KeywordTok{legend}\NormalTok{(}\StringTok{"topright"}\NormalTok{,}\KeywordTok{c}\NormalTok{(}\StringTok{"LO2=-0.421"}\NormalTok{, }\StringTok{"LO2=1.45"}\NormalTok{, }\StringTok{"LO2=2.44"}\NormalTok{,}\StringTok{"LO2=4.01"}\NormalTok{), }
\DataTypeTok{col=}\KeywordTok{c}\NormalTok{(}\StringTok{'blue'}\NormalTok{,}\StringTok{'red'}\NormalTok{,}\StringTok{'green'}\NormalTok{,}\StringTok{'cyan'}\NormalTok{), }\DataTypeTok{lty=}\KeywordTok{c}\NormalTok{(}\StringTok{"solid"}\NormalTok{))}
\end{Highlighting}
\end{Shaded}

\includegraphics{HW5_629Solutions_files/figure-latex/unnamed-chunk-25-1.pdf}
\textbf{Comment:} The fitted survival curves from LO2 look similar to
non-parametetic survival curves for LO2.group from 5.i.


\end{document}

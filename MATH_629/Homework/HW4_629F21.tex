\documentclass[12pt]{article}
%\input{hw_macros.tex}

\setlength{\topmargin}{-.75in} \addtolength{\textheight}{2.00in}
\setlength{\oddsidemargin}{.00in} \addtolength{\textwidth}{.75in}

\usepackage{amsmath,color,amssymb,graphicx,esdiff}
\usepackage[shortlabels]{enumitem}
\pagestyle{empty}
\usepackage{listings}

\setlength{\parindent}{0in}
\setlist[enumerate]{leftmargin=*}
\begin{document}

{\sc {\bf {\large Homework 4}}
            \hfill {MTH 629, Fall 2021}}
\bigskip

{\bf Due Friday, Nov. 12\textsuperscript{th} by 11:59 pm on Canvas}

\section{Chapter 5}
\vspace{5pt}

The data set Addicts is from a study by Caplehorn et al. (``Methadone Dosage and Retention of Patients in Maintenance Treatment,'' Med. J. Aust., 1991). These data comprise the times indays spent by heroin addicts from entry to departure from one of two methadone clinics. There are two further covariates, namely, prison record and methadone dose, believed to affect the survival times. A listing of the variables is given below:
\begin{enumerate}[i.]
\item Subject ID
\item Clinic (1 or 2)
\item Survival status (0 = censored, 1 = departed from clinic)
\item Survival time in days
\item Prison record (0 = none, 1 = any)
\item Methadone dose (mg/day)
\end{enumerate} 

\textbf{Problems:} 
\begin{enumerate}[1.]
\item Load the data by using the following command in R

 \lstinline{load(``addicts.rda'')}
\item Test the proportion hazard (PH) assumption for the variables clinic, prison and dose.group using the following tests:
\begin{enumerate}[i.] 
\item Plot log-log plots of clinic, prison and dose.group. To form dose.group, stratify dose into three parts (1,2,3). Use the R code \\
\lstinline{addicts$dose.group<-cut(addicts$dose,c(19,50,70,110),labels=c('1','2','3'))}\\
 Add legends to each of your log-log plots. 
\item A GOF test using Schoenfeld Residuals to test each variable. Conduct your test at significance level $\alpha=0.05$. 
\item What conclusion do you reach from the Schoenfeld residuals about the PH assumption? 
\end{enumerate} 
\end{enumerate} 


\end{document}  
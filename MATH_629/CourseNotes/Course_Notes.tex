\documentclass{article}
\usepackage{amsmath, amsfonts}
\usepackage{enumerate}
\usepackage{subfig}
\newcommand{\bn}{\begin{enumerate}[1)]}
\newcommand{\bni}{\begin{enumerate}[i)]}
\newcommand{\en}{\end{enumerate}}
\newcommand{\im}{\item}
\newcommand{\CPT}[1]{\large{\textbf{CHAPTER #1}}}
\newcommand{\ir}[1]{\textbf{Remark #1}}
\newcommand{\ith}[1]{\textbf{Theorem #1}}
\newcommand{\idf}[1]{\textbf{Definition #1}}
\newcommand{\iex}[1]{\textbf{Example #1}}
\newcommand{\NTS}{\textbf{Note to self:}}
\newcommand{\prbm}[1]{\textbf{Problem #1}}
\setlength\parindent{0pt}
\setlength{\parskip}{\baselineskip}%
\begin{document}
\CPT{1}

\idf{1.1} \textbf{Survival analysis} is a collection of statistical procedures for data analysis for which the outcome variable of interest is time, $T$, until an event occurs.

\ir{1.1}
\bn
\im $T$ is a random variable with positive support. Examples are time until death, time until machine breaks, length of road travelled until have a flat tire, etc. We will use the phrases ``time until event'' and ``survival time'' interchangeably.
\im There are extensions and variations of survival analysis including \textbf{recurrent events} and \textbf{competing risks}.
\im Example of recurrent events - multiple hospitalizations. Example of competing risks - death on dialysis and receiving a kidney transplant.
\en

\idf{1.2} \textbf{Censoring} is when you have some, but not all, information about the time until event for some or all of your observations. There are three types of censoring.

We will only deal with right censored data in this course. This is the most common type of censoring.


\ir{1.2} We use the following notation
\bn
\item $T$ is the population random variable that represents time until event. 
\item $t$ is the observed value of $T$.
\item $d$ is 0 or 1. If $d=0$ there is censoring and if $d=1$ an event (failure) occurred. $\delta$ is often used instead of $d$. 
\en

\idf{1.3} The \textbf{survival function} gives the probability that the event will occur after a time $t$. That is,

\[
S(t) = P(T>t) \text{ for } t \in \mathbb{R}. 
\]

Note: 1) The survival function always tends to 1 as t tends to infinity, i.e.,
\[
\lim_{t\rightarrow\infty} S(t) = 1
\]
and 2) for positive random variable $T$ we have $S(t) = 0$ for $t \le 0$.

\NTS Draw a smooth survival curve and draw a step function which is often used to estimate one (see page 10).

\idf{1.4} The \textbf{hazard function} is the instantaneous rate of change (i.e. the derivative) of the probability that the event will occur given that it has not occurred up to that time.
\[
h(t) = \lim_{\Delta \rightarrow 0} \frac{P(t \le T < t + \Delta | T \ge t)}{\Delta} = \frac{d}{dt} [S(t|T \ge T)]
\]

\CPT{2}   

\CPT{3}

\ith{3.1} Let $LR_F$ be the value of the log-likelihood for a fitted Cox-Proportional Hazard model
\[ h(t;X) = h_0(t) \exp(\sum_i^p \hat{B}_iX_i) \].
Let $LR_R$ be the value of the log-likelihood for a fitted Cox-Proportional Hazard model
\[ h(t;X) = h_0(t) \exp(\sum_i^q \hat{B}_iX_i) \]
where $q<p$.
Then for a ``large sample'' we have the following approximate distribution
\[
-2(LR_R - LR_F) \sim \chi^2(p-q)
\]


\end{document} 
\documentclass{article}
\usepackage{amsmath}
\usepackage{enumerate}
\newcommand{\bn}{\begin{enumerate}[1)]}
\newcommand{\bni}{\begin{enumerate}[i)]}
\newcommand{\en}{\end{enumerate}}
\newcommand{\im}{\item}
\newcommand{\CPT}[1]{\large{\textbf{CHAPTER #1}}}
\newcommand{\ir}[1]{\textbf{Remark #1}}
\newcommand{\ith}[1]{\textbf{Theorem #1}}
\newcommand{\idf}[1]{\textbf{Definition #1}}
\newcommand{\iex}[1]{\textbf{Example #1}}
\newcommand{\NTS}{\textbf{Note to self:}}
\newcommand{\prbm}[1]{\textbf{Problem #1}}
\setlength\parindent{0pt}
\setlength{\parskip}{\baselineskip}%
\begin{document}
\CPT{1}

\idf{1.1} \textbf{Survival analysis} is a collection of statistical procedures for data analysis for which the outcome variable of interest is time, $T$, until an event occurs.

\ir{1.1}
\bn
\im $T$ is a random variable with positive support. Examples are time until death, time until machine breaks, length of road travelled until have a flat tire, etc.
\im There are extensions and variations of survival analysis including \textbf{recurrent events} and \textbf{competing risks}.
\item Example of recurrent events - multiple hospitalizations. Example of competing risks - death on dialysis and receiving a kidney transplant.
\en
\CPT{3}

\ith{3.1} Let $LR_F$ be the value of the log-likelihood for a fitted Cox-Proportional Hazard model
\[ h(t;X) = h_0(t) \exp(\sum_i^p \hat{B}_iX_i) \].
Let $LR_R$ be the value of the log-likelihood for a fitted Cox-Proportional Hazard model
\[ h(t;X) = h_0(t) \exp(\sum_i^q \hat{B}_iX_i) \]
where $q<p$.
Then for a ``large sample'' we have the following approximate distribution
\[
-2(LR_R - LR_F) \sim \chi^2(p-q)
\]


\end{document} 
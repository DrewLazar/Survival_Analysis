\documentclass{article}
\usepackage{amsmath, amsfonts}
\usepackage{enumerate}
\usepackage{subfig}
\newcommand{\bn}{\begin{enumerate}[1)]}
\newcommand{\bni}{\begin{enumerate}[i)]}
\newcommand{\en}{\end{enumerate}}
\newcommand{\im}{\item}
\newcommand{\CPT}[1]{\large{\textbf{CHAPTER #1}}}
\newcommand{\ir}[1]{\textbf{Remark #1}}
\newcommand{\ith}[1]{\textbf{Theorem #1}}
\newcommand{\idf}[1]{\textbf{Definition #1}}
\newcommand{\iex}[1]{\textbf{Example #1}}
\newcommand{\NTS}{\textbf{Note to self:}}
\newcommand{\prbm}[1]{\textbf{Problem #1}}
\setlength\parindent{0pt}
\setlength{\parskip}{\baselineskip}%
\begin{document}

\CPT{3}

\idf{3.2} If for any two specifications of covariates $\mathbf{X}$ and $\mathbf{X}^*$, and for any $t>0$ we have
\[
\frac{h(t,\mathbf{X})}{h(t,\mathbf{X}^*)} = \theta \text{ for some constant } \theta > 0.
\]
then our population satisfies the \textbf{proportional hazards assumption}.

\ith{3.1} If a population satisfies the Cox PH assumption then it satisfies the proportional hazards assumption. \\
Assume the Cox PH assumption is satisfied then
\[
\frac{h(t,X)}{h(t,X^*)}=\exp(\beta'(X-X^*)) = \exp\left(\sum_{i=1}^p  \beta_i(X_i-X_i^*)\right)
\]

\ir{3.2}
\NTS \, Show 2.,  3. Time-independent: sex, race, age at start of study, smoking status at start of study. Can change, of course. With time-independent covariate you are accounting for the effect at the beginning of the study only.


\newpage
\prbm{3.1}
\begin{enumerate}
\item Model 1: $h(t;TR) = h_{0,1}(t)\exp(\hat{\beta}_1 TR) = h_{0,1}(t)\exp(1.5721 TR)$
\item Model 2: $h(t;TR,logWBC) = h_{0,2}(t)\exp(\hat{\beta}_1 \text{TR} + \hat{\beta}_2 \text{logWBC} ) = h_{0,2}(t)\exp( 1.3861 \text{TR} + 1.6909 \text{logWBC})$
\item Model 3: $h(t;TR,logWBC) = h_{0,3}(t)\exp(\hat{\beta}_1 \text{TR} + \hat{\beta}_2 \text{logWBC} + \hat{\beta}_3 \text{TR}*\text{logWBC}) = h_{0,3}(t)\exp( 2.3749 \text{TR} + 1.8724 \text{logWBC} - 0.3175 \text{TR}*\text{logWBC})$
\item From TR=0 to TR =1 the hazard ratio increases by approximately a factor of $\hat{HR}(TR)=\exp(1.5721) = 4.817.$ \\
    95\% CI for $\beta_1: \hat{\beta_1} \pm z_{\alpha/2} 0.4125$ which gives (0.7638, 2.3804). \\
    95\% CI for $HR(TR): (\exp(0.7638),\exp(1.6529)) = (2.1464, 10.8903)$ \\
\end{enumerate}
Is $\hat{\beta_1}$ significant at significance level $\alpha=0.05$? Since $0 \notin 95\%$ CI for $\beta_1$, yes!
More formally, we test $H_0: \beta=0$ vs. $H_a: \beta_1 \neq 0$. Under $H_0$,
\[ \frac{\hat{\beta}_1 -\beta_1}{SE({\hat{\beta}_1})} =  \frac{\hat{\beta}_1}{SE({\beta_1})}  \text{ approx } N(0,1)
\]
We observe
\[
z = \frac{\hat{\beta}_1}{SE{\beta_1}} = \frac{1.5721}{0.4124} = 3.812
\]
p-value = $2*P(Z>|z|) = 2*P(z>3.812) = 0.000138$. Since p-value $< \alpha=0.05$ we reject $H_0$. \\
Similarity, we can get the coefficients and estimates for model 2 and model 3. \\
If we include the interaction term then the approximate effect of treatment depends on logWBC. Specifically, from $TR=0$ to $TR=1$ the hazard ratio changes by approximately a factor of
\[\exp(\hat{\beta_1}+ \hat{\beta}_3 \text{logWBC}) = \exp(2.3749 - 0.3175\text{logWBC}) \]

Note the interaction coefficient has a large p-value of 0.546 and is \textit{not significant } for any $\alpha > 0.546$. \\
Thus, we might exclude the interaction term and decide our ``best'' model is model 2. This is assuming that our COX PH assumption is met (in chapter 4 we will consider this). \\

\ir{3.6}
Large sample C.I.'s for $\hat{HR}$. \\

Assume we have the term $X_i\beta_i$ in our Cox PH model. A ``large sample'' C.I. for a parameter $\beta_i$ is:
\[
\hat{\beta_i} \pm z_{\alpha/2} SE(\hat{\beta_i})
\]
Thus, without interaction $\hat{HR} = \exp(\hat{\beta}_i)$ and a ``large sample'' C.I. for $HR$ is
\[
\exp(\hat{\beta_i} \pm z_{\alpha/2} SE(\hat{\beta}_i))
\]
What if we had interaction? Then
\[
\hat{HR} = \exp(\hat{\beta}_i + \sum \hat{\beta_j}X_j)
\]
where $X_j's (j \neq i)$ are all the covariates that $X_i$ interacts with. What is the CI for $\ell = \beta_i + \sum \beta_jX_j$ when we have interaction?
\[
\hat{\ell} = \hat{\beta}_i + \sum \hat{\beta}_jX_j
\]
and the $100(1-\alpha)\%$ CI for $\ell$ is
\[
\hat{\ell} \pm z_{\alpha/2} SE(\hat{\ell})
\]
Assume we have one $X_j$ (for simplicity).
Then
\begin{align*}
& SE(\hat{\ell}) = \sqrt{var(\hat{\ell})} = \sqrt{cov(\hat{\ell},\hat{\ell})} \\
&\sqrt{cov(\hat{\beta}_i+ \hat{\beta}_jX_j,\hat{\beta}_i+ \hat{\beta}_jX_j)} = \\
&\sqrt{var(\hat{\beta}_i) + 2X_iX_j cov(\hat{\beta}_i,\hat{\beta}_j) + X_j^2 var{\hat{\beta}_j}}
\end{align*}

Then a $100(1-\alpha)\%$ CI for $HR$ is
\[
\exp(\hat{\ell} \pm z_{\alpha/2} SE(\hat{\ell}))
\]
Example for Remission Data - Hazard ratio using mean of logWBC under interaction with TR.
\[
\hat{HR} = \exp(\hat{\beta_1}+\overline{logWBC}\hat{\beta}_3)
\]
and $95\%$ C.I. for $\hat{HR}$ is
\[
\exp\left(\hat{\beta_1}+\overline{logWBC}\hat{\beta}_3 \pm 1.96 \sqrt{var(\hat{\beta}_1)+2\overline{logWBC}cov(\hat{\beta}_1,\hat{\beta}_3)+\overline{logWBC}^2var(\hat{\beta}_3)}\right)
\]

\ith{3.2}
Proof:
By \ith{1.1} we have the relation
\[
S(t;X) = e^{\int_0^t h(u;X)du}
\]
Thus, in this case we have
\begin{align*}
S(t;X) & =  e^{\int_0^t h(u;X)du} = e^{\int_0^t h_0(u)e^{X'\beta}du} \\
& = (e^{\int_0^t h_0(u)du})^{e^{X'\beta}} = S_0(t)^{e^{X'\beta}}
\end{align*}


Now once we have estimates, $\hat{\beta}$ we have
\[
\hat{S}(t,X) = \hat{S}_0(t)^{e^{X'(\hat{\beta})}}
\]
But how do we get $\hat{S}_0(t)$? We let the computer do it, with a Kaplan-Meirer like process (but more difficult) we are not going to examine this estimation in this class.


\ir{3.7}
Assume a random variable $X$ has a probability distribution $f(x;\beta_1,\ldots,\beta_p)$ that depends on parameters $\beta_1,\ldots,\beta_p$. If we then observe a sample from $X$, $x_1,\ldots,x_n$. Then the likelihood function is
\[
(*) L(\beta_1,\ldots,\beta_n) = g(x_1,\ldots,x_n;\beta_1,\ldots,\beta_n)
\]
where $g$ the joint probability of the sample $X_1,\ldots,X_n$.

If $X_1,\ldots,X_n$ is a random sample then $g(x_1,\ldots,x_n;\beta_1,\ldots,\beta_n) = f(x_1;\beta_1,\ldots,\beta_p) \cdots  f(x_n;\beta_1,\ldots,\beta_p)$.

The values of $\beta_1,\ldots,\beta_n$ that maximize (*) with respect to given observations are known as maximimum likelihood estimates (MLE's) and are denoted $\hat{\beta}_1,\ldots,\hat{\beta}_n$, respectively.

Because $0<L(\beta_1,\ldots,\beta_n) \leq 1$ we can take it's natural log
\[
\ell(\beta_1,\ldots,\beta_n) = \ln[L(\beta_1,\ldots,\beta_n)]
\]
and because $\ln$ is a strictly increasing function we can maximize $L$ by maximizing $\ell$. $\ell(\beta_1,\ldots,\beta_n)$ is known as a log-likelihood function.

If $X$ is discrete then
\begin{align*}
&L(\beta_1,\ldots,\beta_n) = g(x_1,\ldots,x_n;\beta_1,\ldots,\beta_n) = \\
&P[(X_1=x_1) \cap (X_2=x_2) \ldots \cap X_n = x_n;\beta_1,\ldots,\beta_n)
\end{align*}
so by maximizing $L$ we are finding the parameters that are most likely to have produced the observations $x_1,\ldots,x_n$.
To maximize $\ell$ we typically solve the \textbf{score equations}
\begin{align*}
\frac{\partial \ell}{\partial \beta_1}=0 \\
\ldots \\
\frac{\partial \ell}{\partial \beta_p}=0
\end{align*}

\prbm{3.3}
R returns the log likelihood of the fully reduced model fit with a single, constant main effect.
To test for $\beta_1=0$ in
\[ h(t;X) = h_0(t)\exp(\beta_1 TR)
\]
using LRT we have $-2(LR_R-LR_F)=-2(-93.1842-(-85.00842)) = 16.3515$


To test for $\beta_2=0$ in 
\[
h(t;X) = h_0(t)\exp(\beta_1 TR + \beta_2 logWBC) 
\]
using LRT we have $-2(\ln(L_R)-\ln(L_F)) = -2((-85.00842)-(69.82810))=

\end{document} 
\documentclass{article}
\usepackage{amsmath, amsfonts}
\usepackage{enumerate}
\usepackage{subfig}
\newcommand{\bn}{\begin{enumerate}[i)]}
\newcommand{\en}{\end{enumerate}}
\newcommand{\im}{\item}
\newcommand{\CPT}[1]{\large{\textbf{CHAPTER #1}}}
\newcommand{\ir}[1]{\textbf{Remark #1}}
\newcommand{\ith}[1]{\textbf{Theorem #1}}
\newcommand{\idf}[1]{\textbf{Definition #1}}
\newcommand{\iex}[1]{\textbf{Example #1}}

\setlength\parindent{0pt}
\setlength{\parskip}{\baselineskip}%
\begin{document}
\CPT{1}

\idf{1.1} \textbf{Survival analysis} is a collection of statistical procedures for data analysis for which the outcome variable of interest is time, $T$, until an event occurs.

\ir{1.1}
\bn
\im $T$ is a random variable with positive support. Examples are time until death, time until machine breaks, length of road travelled until have a flat tire, etc. We will use the phrases ``time until event'' and ``survival time'' interchangeably.
\im There are extensions and variations of survival analysis including \textbf{recurrent events} and \textbf{competing risks}.
\im Example of recurrent events - multiple hospitalizations. Example of competing risks - death on dialysis and receiving a kidney transplant.
\en

\idf{1.2} \textbf{Censoring} is when you have some, but not all, information about the time until event for some or all of your observations. There are three types of censoring.

We will mostly deal with right censored data in this course. This is the most common type of censoring.


\ir{1.2} We use the following notation
\bn
\item $T$ is the population random variable that represents time until event.
\item $t$ is the observed value of $T$.
\item $d$ is 0 or 1. If $d=0$ there is censoring and if $d=1$ an event (failure) occurred. $\delta$ is often used instead of $d$.
\en

\idf{1.3} The \textbf{survival function} gives the probability that the event will occur after a time $t$. That is,

\[
S(t) = P(T>t) \text{ for } t \in \mathbb{R}.
\]

Note: 1) The survival function always tends to 0 as t tends to infinity, i.e.,
\[
\lim_{t\rightarrow\infty} S(t) = 0
\]
and 2) for positive random variable $T$ we have $S(t) = 1$ for $t \le 0$.

\NTS \, Draw a smooth survival curve and draw a step function which is often used to estimate one (see page 10).

\idf{1.4} The \textbf{hazard function} is the instantaneous rate of change of the probability that the event will occur given that it has not occurred up to that time.
\[
h(t) = \lim_{\Delta \rightarrow 0} \frac{P(t \le T < t + \Delta t | T \ge t)}{\Delta t}
\]
Note: 1) The hazard function is a rate not a probability, 2) the hazard can be any number from 0 to infinity, i.e, $0\ge h(t) \ge \infty$, 3) the hazard depends on the unit of measurement of time. For example, all other things being equal, if time is measured in minutes as opposed to seconds then the hazard is increased by a factor of 60.

\ith{1.1} The hazard function and the survival function of a distribution have the following relationship

\[ S(t) = \exp\left[-\int_0^t h(u) du\right] \text{ and } h(t) = -\frac{d S(t)/dt}{S(t)}
\]
Proof:
\begin{align*} h(t) & = \lim_{\Delta \rightarrow 0} \frac{P(t \le T < t + \Delta t)} {P(T \ge t) \Delta t} \\
& = \frac{1}{S(t)} \lim_{\Delta \rightarrow 0} \frac{P(t \le T < t + \Delta t)} {\Delta t} \\
& = \frac{dF(t)/dt}{S(t)} = -\frac{d S(t)/dt}{S(t)}.
\end{align*}
Now,
\begin{align*}
\exp\left[-\int_0^t h(u) du\right] & = \exp\left[-\int_0^t \left(-\frac{dS(u)/du}{S(u)} \right)du\right] \\
&= \exp\left[\ln (S(u)) |_0^t \right] = \exp\left[\ln(S(t)) - \ln(S(0))\right] \\
& = \exp\left[\ln(S(t))\right] = S(t).
\end{align*}

\ir{1.2} The first way of representing the data is how we will store our data in a data frame in R. The second way is a ``survival object'' that we create in R from the data frame. From the third way we will be able to estimate Kaplan-Meirer survival curves from each of our groups (RX=0 and RX=1). The fourth way is a generalization of the first way that allows for age at follow up (rather than time), time-dependent variables and recurrent events. Consider the third way:
\begin{enumerate}[ ]
\item The leftmost column $(t_{(f)})$ gives all of the times at which the event occurs for $f=1,\ldots,n$ plus $t_{(0)}$ at the start of the study.
\item The second column ($m_f$) gives the number of events that occur at each of the times.
\item The third column ($q_f$) gives the number of observations that were censored in time $[t_f,t_{f+1})$ for $f=0,\ldots,n$.
\item The last column $R(t_f)$ gives the number ``at risk'' at each time. For example, at time $t_{(1)}=6$, $m_{1}=3$ had the event, $q_{1}=1$ was censored in $[t_{(1)}=6, t_{(2)}=7)$ and
    \[
    R(2)=R(1)-m_{(1)}-q_{(1)}=21-3-1=17
    \]
     were ``at risk'' at time $t_{(2)}=7$.
\end{enumerate}
This way of presenting data, helps us estimate probabilities, at each time $t_{f}$ that a subject will ``survive'' beyond time $t_{(f)}$ (i.e. not have the event before and including time $t_{(f)}$) given that they have ``survived'' at least until time $t_{(f)}$ (i.e. not had the event before time $t_{(f)}$)  For example, at $t_{(2)}$ we estimate this probability as
\[ \widehat{P(T>t_{(2)}| T \ge t_{(2)})} =  \dfrac{R(2)-m_2}{R{(2)}} = \dfrac{17-1}{17} = \dfrac{16}{17}
\]
This will allow us in chapter 2 to estimate overall survival probabilities at times $t_{(f)}, f=1,\ldots,n$ and thus estimate survival curves as step functions.

\idf{1.5}
\begin{enumerate}
\item The \textbf{average observed survival time }(ignoring censoring) is denoted by $\bar{T}$.
\item The \textbf{average hazard rate} is the total number of failures divided by the total observed survival time. It is denoted by $\bar{h}$.
\end{enumerate}
 For group 1 in the Remission data we have
\[ \bar{T}_1 = \frac{6 + 6 + 6 + \ldots + 35}{21} \approx 17.1 \text { and } \bar{h}_1 \approx \frac{9}{6 + 6 + 6 + \ldots + 35} = .025
\]
For group 2 in the Remission data we have
\[ \bar{T}_2 = \frac{1 + 1 + 1 + \ldots + 23}{21} \approx 8.6 \text { and } \bar{h}_2 = \frac{21}{1 + 1 + 1 + \ldots + 23} \approx .115
\]

Note for group 2, without censoring we have $\bar{h}_2 = 1/\bar{T}_2$. These are crude overall measures. If there was no censoring in group 1 and censoring in group 2 then $\bar{T}_1$ and $\bar{T}_2$ don't change and can be misleading. Instead, we want to estimate survival probabilities at particular times and for certain levels of covariates. That will be the subject of the rest of the book.


\NTS \, Point out Kaplan Meirer survival curves. Point out little difference in treatments early on but that Group 1 treatment is more effective as time goes on. Also, point out that you can get another overall measure the \textbf{observed median time}.

\idf{1.6} An \textbf{exposure variable} is the variable that you are interested in establishing at least a correlative if not causative relationship with the survival variable. \textbf{Secondary variable} is another variable in the data set that can also have an effect on survival times.

\ir{1.3} `Untangling' the effect of exposure vs secondary variables on survival times can thus be difficult. In this regard, there are two issues to consider, confounding and interaction.

\idf{1.7} \textbf{Confounding } is the extent to which the effect of the change of a variable (usually the exposure variable) on survival experience is caused by an accompanying change in a secondary variable.

\idf{1.8} \textbf{Interaction} is when the effect of the change in a variable (usually the exposure variable) on survival experience depends on the level of a secondary variable.


\iex{1.1} \\
3 year (group A): $P_A(\text{survive 3 years})= 80/100 = 0.80 \implies P_A(\text{fail 3 years}) = 0.20$ \\ \\
5 year (group A): 40 censored out of 80. Of 40 left in risk set, 5/40 = 0.125 = 12.5\% had the event. By independent censoring, we assume .125*40 = 5 of censored also had the event. All together over 5 years, we assume 20 + 5 + 5 = 30   had the event. Thus, $P_B(\text{survive 5 years})= 60/100 = 0.60 \implies P_A(\text{fail 5 years}) = 0.40.$ \\

3 year (group B): $P_A(\text{survive 3 years})= 60/100 = 0.60 \implies P_A(\text{fail 3 years}) = 0.40$ \\ \\
5 year (group B): 10 censored out of 60. Of 50 left in risk set, 10/50 = 0.20 = 20\% had the event. By independent censoring, we assume .20*10 = 2 of censored also had the event. All together over 5 years, we assume 40 + 10 + 2 = 52 had the event. Thus, $P_B(\text{survive 5 years})= 48/100 = 0.48 \implies P_B(\text{fail 5 years}) = 0.52.$ \\


3 year (All): $P(\text{survive 3 years})= 140/200 = 0.70 \implies P(\text{fail 3 years}) = 0.30$ \\ \\
5 year (All): 50 censored out of 140. Of 90 left in risk set, $15/90 = 0.1\overline{666} = 16.\overline{666}\%$ had the event. By random censoring, we assume $(15/90)*50 = 8.\overline{333}$ of censored also had the event. All together over 5 years, we assume $60 + 15 + 8.\overline{333} = 83.\overline{333}$ had the event. Thus, $P(\text{survive 5 years})= 16.\overline{666}/100 = 0.1\overline{666} \implies P(\text{fail 5 years}) = 0.8\overline{333}$. However, assuming independent censoring, we assume 92 had the event and  $P(\text{survive 5 years})= 8/100 = 0.08 \implies P(\text{fail 5 years}) = 0.92$.

\end{document}

